\chapter[CH:CT]{Diagrammes de classes détaillés}

Dans cette partie nous présentons différents diagrammes de classes détaillés.
Les classes ainsi que les méthodes ont été pour la plupart déduites directement d'un diagramme de séquence.
Nous continuons à utiliser le code couleur introduit lors des diagrammes de classes métiers.

\startitemize
\item le diagramme de la \in{figure}[DC:BE] correspond au \cu\ «UtiliserBorneEuréka», voir \in{section}[SS:CUD:UBE].
Le diagramme de séquence correspondant est à la \in{figure}[DS:BE].

\item le diagramme de la \in{figure}[DC:AAC] correspond au \cu\ «AjouterArticleCadBot», voir \in{section}[SS:CUD:AAC].
Le diagramme de séquence correspondant est à la \in{figure}[DS:AAC].

\item le diagramme de la \in{figure}[DC:PP] correspond au \cu\ «PasserPortiquePaiement», voir \in{section}[SS:CUD:PP].
Le diagramme de séquence correspondant est à la \in{figure}[DS:PP].

\item le diagramme de la \in{figure}[DC:AR] correspond au \cu\ «AppliquerRecetteConnectée», voir \in{section}[SS:CUD:AR].
Le diagramme de séquence correspondant est à la \in{figure}[DS:AR1].

% TO BE CONTINUED
\stopitemize

% placer les figures ici
\placefigure[][DC:BE]{Classes détaillées modélisant une borne Euréka}
{\externalfigure[ECLIPSE-LUNA-DS-BORNE-EUREKA/ClassesBorneEurekaMVC.pdf][width=\hsize]}

\placefigure[][DC:AR]{Détail des classes impliquées dans le \cu\ «AppliquerRecetteConnectée»}
{\externalfigure[ECLIPSE-LUNA-DS-APPLIQUER-RECETTE/ClassesAppliquerRecetteMVC.pdf][width=\hsize]}

\placefigure[][DC:AAC]{Détail des classes impliquées dans le \cu\ «AjouterArticleCadBot»}
{\externalfigure[ECLIPSE-LUNA-DS-AJOUTER-ARTICLE-CADBOT/ClassesAjouterArticleCadBot.pdf][width=\hsize]}

\placefigure[][DC:PP]{Détail des classes impliquées dans le \cu\ «PasserPortiquePaiement»}
{\externalfigure[ECLIPSE-LUNA-DS-PASSER-PORTIQUE/ClassesPasserPortiquePaiement.pdf][width=\hsize]}


% TO BE CONTINUED
