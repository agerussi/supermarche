\chapter{Classes métier}

Pour davantage de lisibilité, nous présentons quatre diagrammes de classes différents:
\startitemize
\item le diagramme de la \in{figure}[CLASSES:GENERAL] donne une vue générale de haut niveau, dont les différentes classes sont ensuite détaillées séparément;
\item la partie concernant les personnes est présentée en \in{figure}[CLASSES:INDIVIDU];
\item celle représentant le matériel est à la \in{figure}[CLASSES:MATERIEL];
\item enfin, la partie concernant les produits et les concepts associés (liste connectée, recette connectée, article générique\dots) sont à la \in{figure}[CLASSES:PRODUIT].
\stopitemize

\placefigure[][CLASSES:GENERAL]
{classes de haut niveau dans le système}
{\externalfigure[ECLIPSE-LUNA-CLASSES-METIER/General.pdf][width=\hsize]}

\placefigure[][CLASSES:INDIVIDU]
{diagramme des classes concernant les personnes}
{\externalfigure[ECLIPSE-LUNA-CLASSES-METIER/Individu.pdf][width=\hsize]}

\placefigure[][CLASSES:MATERIEL]
{diagramme des classes concernant le matériel}
{\externalfigure[ECLIPSE-LUNA-CLASSES-METIER/Materiel.pdf][width=\hsize]}

\placefigure[][CLASSES:PRODUIT]
{diagramme des classes concernant les produits}
{\externalfigure[ECLIPSE-LUNA-CLASSES-METIER/Produit.pdf][width=\hsize]}

