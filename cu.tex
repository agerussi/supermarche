\chapter{Cas d'utilisation}
Nous avons dessiné deux diagrammes séparés.
En effet nous nous sommes focalisés essentiellement sur les clients et le diagramme correspondant, présenté à la \in{figure}[CUC], est relativement riche.
Nous avons donc dessiné un deuxième diagramme pour les cas d'utilisation concernant le personnel du futurmarket, \in{figure}[CUP].
\placefigure[][CUC]{Cas d'utilisation des clients}{\externalfigure[ECLIPSE/diagrammeCU.pdf][width=\hsize]}
\placefigure[][CUP]{Cas d'utilisation du personnel}{\externalfigure[ECLIPSE/CUPersonnel.pdf][width=\hsize]}

\section{Cas d'utilisation principaux}

\subsection{Choisir son panier}
Ce cas d'utilisation regroupe tous les moyens à la disposition du client pour établir son panier d'achats.

\subsection{Récupérer les articles}
Ce cas d'utilisation regroupe tous les moyens à la disposition du client pour récupérer les articles qu'il désire acheter, à partir d'un panier déjà établi ou lors d'un achat spontané.

\subsection{Passer en caisse}
Ce cas d'utilisation englobe le calcul du montant total des articles prélevés en magasin et le paiement proprement dit.

\subsection{Payer}
Désigne le paiement proprement dit, quel que soit le moyen utilisé à cette fin.

\subsection{Suivi emplacement des articles}
Ce \cu\ englobe l'ensemble des actions destinées à assurer la correspondance entre le ou les emplacement(s) effectif(s) des articles dans les rayons, et le ou les emplacement(s) théoriques enregistrés dans la base de données.

\subsection{Gestion stocks}
Ce \cu\ regroupe les actions nécessaires au suivi de la disponibilité de chaque article, par exemple l'estimation de la date de rupture de stocks, les commandes fournisseurs, \etc

\subsection{Gestion du personnel}
Ce \cu\ regroupe tous les aspects liés à la gestion du personnel: plannings, pointage, périodes de non-disponibilité, affectations à divers postes, \etc

\section{Cas d'utilisation secondaires}

\subsection{Panier recettes}
Dans ce cas d'utilisation, le client utilise les recettes pour s'aider à constituer son panier.

\subsection{Panier budget}
Dans ce cas d'utilisation, le client gère son panier selon des contraintes de budget, en utilisant les services proposés à cet effet par le système.

\subsection{Panier amazon}
Dans ce cas d'utilisation, le client gère son panier par lui-même, en ligne, à l'aide de ses paniers pré-établis, des associations de produits proposés spontanément par le système.

\subsection{Panier en direct}
Ce cas d'utilisation désigne la constitution du panier en temps réel dans le magasin, à l'aide notamment des possibilités offertes par la markettab.

\subsection{Cadbot}
Ce cas d'utilisation décrit l'utilisation du cadbot en temps que support pour la récupération effective des articles en magasin.

\subsection{Markettab}
Ce cas d'utilisation décrit l'utilisation d'une markettab pour choisir son panier, trouver un article, ou être guidé dans le magasin.

\subsection{Euréka}
Dans ce cas d'utilisation, le client interroge une borne euréka pour localiser un produit ou s'informer de sa disponibilité.

\subsection{Portique de paiement}
Dans ce cas d'utilisation, le client utilise un cadbot et va procéder au passage en caisse par un portique prévu à cet effet. 

\subsection{Caisse hôtesse}
Ce cas décrit un passage classique dans une caisse comportant une hôtesse, incluant un débalage, scanning et remballage des articles du caddie.

\subsection{Paiement téléphone}
TODO

\subsection{Paiement espèces}
Ce cas d'utilisation désigne l'utilisation de monnaie papier ou métallique pour payer ses achats, soit à une caisse classique, soit lors d'une livraison ou d'un retrait au drive.

\subsection{Paiement en ligne}
Ce cas d'utilisation désigne un paiement effectif de chez soi, à partir du site du futurmarket.

\subsection{Paiement carte}
Ce cas d'utilisation désigne l'utilisation d'une carte (banquaire ou carte magasin) dans un terminal de paiement (caisse, portique, point drive, livraison).

\subsection{Drive}
Dans ce cas d'utilisation, le client se fait préparer un panier préalablement constitué en ligne et va le récupérer lui-même au point de retrait drive du futurmarket.
Le paiement peut avoir lieu en ligne ou sur place lors du retrait.

\subsection{Livraison}
Dans ce cas d'utilisation, le client se fait livrer un panier préalablement constitué en ligne, directement à son domicile, moyennant un coût de livraison.
Le paiement peut avoir lieu en ligne ou lors de la livraison.

\subsection{Placement par markettab}
Dans ce \cu, un personnel du magasin (généralement un magasinier) se sert d'une markettab pour gérer l'emplacement d'un produit dans la base de données.
Il peut supprimer des emplacements connus, ou en rajouter de nouveaux.

\subsection{Pointage temps de travail}
Ce \cu\ englobe les différents scénarios inhérents à la mesure du temps de présence effectif du personnel.
L'ensemble du personnel doit notamment «pointer» grâce à son badge rfid.
