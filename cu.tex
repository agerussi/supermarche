\title{Cas d'utilisation}

Le diagramme UML des cas d'utilisation est présenté à la \in{figure}[CU].
\placefigure[][CU]{Diagramme des cas d'utilisation}{\externalfigure[ECLIPSE/diagrammeCU.pdf][width=\hsize]}

\section{Cas d'utilisation principaux}

\subsection{Choisir son panier}
Ce cas d'utilisation regroupe tous les moyens à la disposition du client pour établir son panier d'achats.

\subsection{Récupérer les articles}
Ce cas d'utilisation regroupe tous les moyens à la disposition du client pour récupérer les articles qu'il désire acheter, à partir d'un panier déjà établi ou lors d'un achat spontané.

\subsection{Passer en caisse}
Ce cas d'utilisation englobe le calcul du montant total des articles prélevés en magasin et le paiement proprement dit.

\subsection{Payer}
Désigne le paiement proprement dit, quel que soit le moyen utilisé à cette fin.

\section{Cas d'utilisation secondaires}

\subsection{Panier recettes}
Dans ce cas d'utilisation, le client utilise les recettes pour s'aider à constituer son panier.

\subsection{Panier budget}
Dans ce cas d'utilisation, le client gère son panier selon des contraintes de budget, en utilisant les services proposés à cet effet par le système.

\subsection{Panier amazon}
Dans ce cas d'utilisation, le client gère son panier par lui-même, en ligne, à l'aide de ses paniers pré-établis, des associations de produits proposés spontanément par le système.

\subsection{Panier en direct}
Ce cas d'utilisation désigne la constitution du panier en temps réel dans le magasin, à l'aide notamment des possibilités offertes par la markettab.

\subsection{Cadbot}
Ce cas d'utilisation décrit l'utilisation du cadbot en temps que support pour la récupération effective des articles en magasin.

\subsection{Markettab}
Ce cas d'utilisation décrit l'utilisation d'une markettab pour choisir son panier, trouver un article, ou être guidé dans le magasin.

\subsection{Euréka}
Dans ce cas d'utilisation, le client interroge une borne euréka pour localiser un produit ou s'informer de sa disponibilité.

\subsection{Portique de paiement}
Dans ce cas d'utilisation, le client utilise un cadbot et va procéder au passage en caisse par un portique prévu à cet effet. 

\subsection{Caisse hôtesse}
Ce cas décrit un passage classique dans une caisse comportant une hôtesse, incluant un débalage, scanning et remballage des articles du caddie.

\subsection{Paiement téléphone}
TODO

\subsection{Paiement espèces}
Ce cas d'utilisation désigne l'utilisation de monnaie papier ou métallique pour payer ses achats, soit à une caisse classique, soit lors d'une livraison ou d'un retrait au drive.

\subsection{Paiement en ligne}
Ce cas d'utilisation désigne un paiement effectif de chez soi, à partir du site du futurmarket.

\subsection{Paiement carte}
Ce cas d'utilisation désigne l'utilisation d'une carte (banquaire ou carte magasin) dans un terminal de paiement (caisse, portique, point drive, livraison).

\subsection{drive}
Dans ce cas d'utilisation, le client se fait préparer un panier préalablement constitué en ligne et va le récupérer lui-même au point de retrait drive du futurmarket.
Le paiement peut avoir lieu en ligne ou sur place lors du retrait.

\subsection{livraison}
Dans ce cas d'utilisation, le client se fait livrer un panier préalablement constitué en ligne, directement à son domicile, moyennant un coût de livraison.
Le paiement peut avoir lieu en ligne ou lors de la livraison.
