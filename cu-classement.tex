\TODO
Dans ce qui suit, nous proposons un classement des \cu\ par ordre décroissant de priorité de développement.
Nous n'avons pas repris tous les \cu\ in extenso mais uniquement ceux qui se distinguent des autres par un apport de fonctionnalités décisif ou une difficulté de développement particulière. 
En particulier les CU abstraits dont l'existence éventuelle est directement la conséquence de l'implémentation d'un ou plusieurs \cu\ plus spécialisés n'apparaissent pas ici.\par
Le classement s'effectue à l'aide de deux notes:
\startitemize[n]
\item une note «client» (NC), allant de 1 (peu d'intérêt) à 3 (intérêt majeur), qui juge l'apport du \cu\ dans la satisfaction des clients du \fm\ et/ou de la maîtrise d'ouvrage;
\item une note de «risque» (NR), allant de 1 (peu de risques) à 3 (risques importants), qui juge la probabilité qu'à le \cu\ de d'imposer une remise en question majeure de la structure de l'application, et donc d'invalider des développements antérieurs.
\stopitemize
Le classement final est alors effectué selon la somme de ces deux notes.

\def\NC#1{\sym{\bf NC~#1}}%
\def\NR#1{\sym{\bf NR~#1}}%
\def\startnote{\startitemize[][distance=2em]}
\let\stopnote\stopitemize

\descCU{AjouterArticle}
\startnote
\NC0
\NR0
\stopnote

\descCU{AjouterArticleCadbot}
\startnote
\NC0
\NR0
\stopnote

\descCU{AppliquerContrainteBudget}
\startnote
\NC0
\NR0
\stopnote

\descCU{AppliquerRecetteConnectée}
\startnote
\NC0
\NR0
\stopnote

\descCU{ConsulterPromotions}
\startnote
\NC0
\NR0
\stopnote

\descCU{CréerListeConnectée}
\startnote
\NC0
\NR0
\stopnote

\descCU{DétruireListeConnectée}
\startnote
\NC0
\NR0
\stopnote

\descCU{ModifierListeConnectée}
\startnote
\NC0
\NR0
\stopnote

\descCU{PasserCaisseHôtesse}
\startnote
\NC3 Cette fonctionalité est classique mais vraiment indispensable pour les clients qui désirent ne pas se servir d'un cadbot.
De par les risques de «rejet technologique» des personnes peu versées dans les outils numériques (notamment les personnes âgées), conserver ce mode de passage en caisse reste également important pour la MOA.
\NR1 Peu de risques techniques.
\stopnote

\descCU{PasserPortiquePaiement}
\startnote
\NC0
\NR0
\stopnote

\descCU{PayerEnEspèces}
\startnote
\NC0
\NR0
\stopnote

\descCU{PayerEnLigne}
\startnote
\NC0
\NR0
\stopnote

\descCU{PayerParCarte}
\startnote
\NC0
\NR0
\stopnote

\descCU{PayerParTéléphone}
\startnote
\NC0
\NR0
\stopnote

\descCU{RechercherRecetteConnectée}
\startnote
\NC0
\NR0
\stopnote

\descCU{RetirerArticleCadbot}
\startnote
\NC0
\NR0
\stopnote

\descCU{SauvegarderListeConnectée}
\startnote
\NC0
\NR0
\stopnote

\descCU{SupprimerArticle}
\startnote
\NC0
\NR0
\stopnote

\descCU{UtiliserBorneEuréka}
\startnote
\NC0
\NR0
\stopnote

\descCU{UtiliserGuidageMarkettab}
\startnote
\NC0
\NR0
\stopnote

\descCU{UtiliserListeConnectée}
\startnote
\NC0
\NR0
\stopnote

\descCU{UtiliserModeAutonome}
\startnote
\NC0
\NR0
\stopnote

