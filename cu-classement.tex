\section{Classement des cas d'utilisation}
Dans ce qui suit, nous proposons un classement des \cu\ par ordre décroissant de priorité de développement.
Nous n'avons pas repris tous les \cu\ in extenso mais uniquement ceux qui se distinguent des autres par un apport de fonctionnalités décisif ou une difficulté de développement particulière. 
En particulier les CU abstraits dont l'existence éventuelle est directement la conséquence de l'implémentation d'un ou plusieurs \cu\ plus spécialisés n'apparaissent pas ici.\par
Le classement s'effectue à l'aide de deux notes:
\startitemize[n]
\item une note «client» (NC), allant de 1 (peu d'intérêt) à 5 (intérêt majeur), qui juge l'apport du \cu\ dans la satisfaction des clients du \fm\ et/ou de la maîtrise d'ouvrage;
\item une note de «risque» (NR), allant de 1 (solution connue, éprouvée) à 5 (à la limite de l'expérimental), qui juge la probabilité qu'a le \cu\ d'imposer une remise en question majeure de la structure de l'application, et donc d'invalider des développements antérieurs.
\stopitemize
Le classement final est alors effectué selon la somme de ces deux notes.

\subsection{Analyse et notation des \cu}

\def\NC#1{\sym{\bf NC~#1}}%
\def\NR#1{\sym{\bf NR~#1}}%
\def\startnote{\startitemize[packed][distance=2em]}%
\let\stopnote\stopitemize

\descCU{AppliquerContrainteBudget}
\startnote
\NC2 Cette partie du système d'aide à la constitution de panier n'est pas fondamentale mais peut intéresser une certaine catégorie de clients.
\NR2 Aucun problème technique, mais l'ergonomie et la souplesse des options demande un design bien pensé.
Peu de risques pour le reste du système puisque ce \cu\ en est relativement indépendant.
\stopnote

\descCU{AppliquerRecetteConnectée, RechercherRecetteConnectée}
\startnote
\NC4 Le concept de recette connectée est une nouveauté importante qui séduira probablement les clients.
\NR2 Pas de difficulté particulière une fois que le SGBD des articles du \fm\ est en place.
\stopnote

\descCU{ConsulterPromotions}
\startnote
\NC2 Cette possibilité n'est pas fondamentale mais intéresse classiquement une certaine catégorie de clients.
\NR1 Pas de difficulté.
\stopnote

\descCU{ChoisirArticles}
\startnote
\NC5 De ce \cu\ représente la faculté à constituer des listes d'articles réutilisables, et de manière assistée.
La quasi-totalité des fonctionnalités d'un \fm\ dépendent directement ou indirectement de celui-ci.
\NR3 Peu de difficultés techniques, mais les fonctionnalités à développer vont influencer une grande partie du design du système, donc il est important de le traiter relativement tôt.
\stopnote

\descCU{GérerListeConnectée, UtiliserListeConnectée}
\startnote
\NC4 Ces \cu\ sont importants pour le client puisque les possibilités de drive, de commander, ou encore de réutiliser d'anciennes listes de courses en dépendent.
\NR1 Peu de difficultés techniques.
\stopnote

\descCU{GérerRecettesConnectées}
\startnote
\NC3 Cet ensemble de \cu\ accompagne notamment AppliquerRecetteConnectée, un \cu\ important.
Cependant on peut supposer que dans un premier temps, les recettes connectées soient fournies par le \fm, sans forcément permettre au client d'interagir.
Ce \cu\ est donc d'importance secondaire par rapport à ceux qui concernent l'utilisation effective d'une recette.
\NR2 Pas de difficulté particulière une fois que le SGBD des articles du \fm\ est en place, et en particulier le concept d'article générique.
Ce style de fonctionnalités d'édition/suppression est relativement classique.
\stopnote

\descCU{PasserCaisseHôtesse}
\startnote
\NC5 Cette fonctionnalité est classique mais vraiment indispensable pour les clients qui désirent ne pas se servir d'un cadbot.
De par les risques de «rejet technologique» des personnes peu versées dans les outils numériques (notamment les personnes âgées), conserver ce mode de passage en caisse reste également important pour la MOA.
\NR1 Peu de risques techniques.
\stopnote

\descCU{PasserPortiquePaiement}
\startnote
\NC5 Ce mode de passage en caisse est la dernière étape dans l'utilisation du cadbot.
De ce point de vue, il est fondamental pour le client et la MOA.
\NR2 Peu de difficultés si l'on suppose que le reste système cadbot existe et fonctionne, puisqu'il suffit simplement de faire communiquer le cadbot et un terminal de paiement classique (une solution complètement basique serait de scanner un code barre affiché sur l'écran du cadbot).
\stopnote

\descCU{PayerEnEspèces}
\startnote
\NC3 Mode de paiement classique qu'il est relativement important de conserver, même s'il est en voie de disparition dans les gros \fm.
\NR1 Aucun risque.
\stopnote

\descCU{PayerEnLigne}
\startnote
\NC4 Mode de paiement classique depuis quelques années, attendu par les clients.
\NR2 Il y a des risques de sécurité qu'il faut prendre correctement en compte, mais d'un autre côté des solutions techniques éprouvées existent.
\stopnote

\descCU{PayerParCarte}
\startnote
\NC5 Mode de paiement très majoritaire, absolument indispensable.
\NR1 Aucun risque, des solutions classiques existent.
\stopnote

\descCU{PayerParTéléphone}
\startnote
\NC1 Mode de paiement alternatif, de type buyster. Le téléphone est un intermédiaire entre la banque et le commercant, au même titre que la carte bancaire. 
\NR5 Gros risques technologiques et clients. Les clients sont ils prêts a adopter cette methode de paiement.
\stopnote

\descCU{UtiliserBorneEuréka}
\startnote
\NC3 Ce système de bornes est intéressant pour les clients, sans être fondamental.
\NR3 La difficulté principale est la reconnaissance vocale. 
Des solutions clé en main existent sûrement, mais nous n'avons pas d'expérience dans ce domaine.
Le reste de la solution logicielle est classique une fois le système de SGBD des articles mis en place, et en particulier le système de coordonnées pour leur localisation dans le magasin.
Ce \cu\ comporte finalement peu de risques pour le reste du système car il en est essentiellement indépendant.
\stopnote

\descCU{UtiliserCadbot}
\startnote
\NC5 Le concept de Cadbot est essentiel pour la MOA, car il est l'une des innovations importantes qu'elle destine à ses clients.
\NR5 Nombreux points délicats ou méconnus:
\startitemize
\item conception «physique»: caddie avec balance précise et robuste intégrée, plusieurs lecteurs de codes barres, lecteur de puces RFID, mini-ordinateur de bord;
\item problème de la recharge énergétique du cadbot
\item liaisons de l'ordinateur et des périphériques;
\item méconnaissance des systèmes embarqués en général.
\stopitemize
\stopnote

\descCU{UtiliserGuidageMarkettab}
\startnote
\NC3 L'utilité de cette fonctionnalité dépend du type de client.
Le «pressé» qui suit systématiquement les listes connectées composées par sa femme appréciera.
Les autres, qui aiment flâner et prendre leur temps dans les rayons pour se laisser inspirer et compléter leur liste de courses basique ne l'utiliseront probablement pas.
\NR3 
Si l'on considère que le SGBD des articles est en place (avec la localisation de chacun d'eux), cette fonctionnalité n'est pas techniquement très difficile.
La principale difficulté est d'être capable d'afficher un plan du \fm\ sur lequel est mis en évidence le produit à récupérer.
Cependant de nombreux «détails» au niveau de l'ergonomie doivent être mis au point.
Ce \cu\ est découplé du reste du système, donc le risque en termes d'impact est peu important.
\stopnote

\descCU{UtiliserModeAutonome}
\startnote
\NC1 La robotisation complète du cadbot n'est pas un objectif prioritaire pour la MOA.
Cela peut être intéressant, mais à condition que ce soit vraiment performant, ce qui n'est pas gagné dans ce domaine relativement expérimental. 
D'autre part, il pourrait y avoir un phénomène de rejet de la part des clients vis à vis de ces «robots».
Pousser un caddie classique n'est pas considéré comme un effort important par les clients.
Enfin, l'inclusion d'un moteur (entre autres périphériques) dans le cadbot va rendre son coût plus important: pas sûr qu'il y ait bénéfice en fin de compte.
\NR5 Grands défis technologiques. Méconnaissance des solutions existantes, manque de recul sur la faisabilité et la fiabilité des «robots suiveurs». 
À noter cependant: les fonctions de guidage du cadbot sont complètement découplées du reste du système, donc ce \cu\ ne mettra pas le design en péril et peut être considéré n'importe quand dans le projet.
\stopnote





\subsection{Tableau récapitulatif}

Le tableau \in[CLASSEMENT] reprend les notes précédentes et classe les \cu\ dans l'ordre dans lequel nous envisageons le développement.
\placetable[][CLASSEMENT]{Ordre de développement des \cu}{%
\newcount\tmpcount\tmpcount1\def\Num{\the\tmpcount\global\advance\tmpcount by 1}%
\starttable[|c|r|c|c|c|]
  \HL
  \VL \bf rang\VL \bf \cu\VL \bf Total\VL \bf NC\VL \bf NR\VL\FR \HL
  \VL \Num\VL UtiliserCadbot\VL 10\VL 5\VL 5\VL\MR \HL
  \VL \Num\VL ChoisirArticles\VL 8\VL 5\VL 3\VL\MR \HL
  \VL \Num\VL PasserPortiquePaiement\VL 7\VL 5\VL 2\VL\MR \HL
  \VL \Num\VL AppliquerRecetteConnectée, RechercherRecetteConnectée\VL 6\VL 4\VL 2\VL\MR \HL
  \VL \Num\VL GérerListeConnectée, UtiliserListeConnectée\VL 5\VL 4\VL 1\VL\MR \HL
  \VL \Num\VL UtiliserBorneEuréka\VL 6\VL 3\VL 3\VL\MR \HL
  \VL \Num\VL UtiliserGuidageMarkettab\VL 6\VL 3\VL 3\VL\MR \HL
  \VL \Num\VL PasserCaisseHôtesse\VL 6\VL 5\VL 1\VL\MR \HL
  \VL \Num\VL PayerParCarte\VL 6\VL 5\VL 1\VL\MR \HL
  \VL \Num\VL PayerEnLigne\VL 6\VL 4\VL 2\VL\MR \HL
  \VL \Num\VL GérerRecettesConnectées\VL 5\VL 3\VL 2\VL\MR \HL
  \VL \Num\VL PayerEnEspèces\VL 4\VL 3\VL 1\VL\MR \HL
  \VL \Num\VL AppliquerContrainteBudget\VL 4\VL 2\VL 2\VL\MR \HL
  \VL \Num\VL ConsulterPromotions\VL 3\VL 2\VL 1\VL\MR \HL
  \VL \Num\VL PayerParTéléphone\VL 6\VL 1\VL 5\VL\MR \HL
  \VL N.C\VL \it UtiliserModeAutonome\VL 6\VL 1\VL 5\VL\MR \HL
\stoptable}
    
Cet ordre est influencé par la note globale obtenue par chaque \cu, mais pas uniquement. 
\par
Nous avons remonté notamment GérerListeConnectée car il s'agit d'un concept important de notre application. 
Nous avons également repoussé PayerParTéléphone car il s'avère que d'autres tentatives déjà existantes d'un tel système n'aient rencontré qu'un succès mitigé.
\par
Le cas d'UtiliserModeAutonome est particulier: c'est un \cu\ très compliqué technologiquement, peu risqué pour le système à développer car indépendant, mais cependant susceptible de demander des modifications importantes au niveau matériel (hardware) sur le cadbot selon les solutions envisagées.
Nous l'avons donc mis entre parenthèses et hors classement pour signifier qu'il n'est envisageable de s'y intéresser que si nous pouvons le faire en parallèle du reste des \cu. 
Il faudrait attribuer une petite équipe à la recherche de solutions pendant que le travail principal pourrait démarrer selon l'ordre retenu.
