\section{Journée hotesse de caisse}

Mariline arrive au supermarché a 08h45. ELle commence son poste du matin a 09h00. Elle passe au vestiaire pour se changer. ELle prend sa badge RFID pour pointer son heure d'arrivée(RFID faible distance), la tablette fixée au mur enregistre l'heure et lui indique la caisse ou il est prévue qu'elle travaille(no 58).Elle passe prendre le tiroir caisse no58.Arrivée a la caisse no 58, mariline s'installe et pose son tag RFID pour deverouiller le logiciel de la caisse. ELle peut maintenant inserer sa caisse monetique dans l'appareil. Elle est maintenant prete pour recevoir les clients. Le premier client arrive, il pose ses courses sur le tapis roulant. Mariline prend les produits un par un pour les scanner. passent alors, la farine, le sucre, la levure, la confiture, du sirop d'erable, et du cidre. mariline se dit qu'elle aussi elle ferait bien des crepes ce soir. Lorsque les produits sont scannés, elle demande au client s'il a sa carte client. Ce dernier lui tend, elle peut ainsi enregistrer la panier client sur le compte du client.Toutes les heures, une collegue vient prelever les billets par lots de 500 euros. La collegue a un badge speciale qui valide le retrait des lots de 500 euros dans le logiciel de caisse. Après avoir traités 94 clients ce jour, elle quitte son poste sans oublier de repartir avec le tiroir caisse et de verrouiller le logiciel de caisse. Mariline depose alors le tiroir caisse dans le local securisé, se change et repart chez elle en badgeant avant de partir.
