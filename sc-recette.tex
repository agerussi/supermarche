\section{Recette}

Alex se connecte sur le site du distributeur et choisit une liste de course vierge.
Alex choisit le menu Recettes.
Alex selectionne autant de recette qu'il prévoit de repas ainsi que les options associés ( végétarien,kasher, halal, nombre de personnes, temps de préparation, ....)
Alex clique sur 'composition panier'
Alex complete sa liste.
Alex ajoute à la main des produits ménagers dont il a besoin à sa liste
Alex sauvegarde sa liste.
Alex choisit ou non d'imprimer les recettes associées.
Alex se rend chez son distributeur.
\startitemize
\item {\bf Variante 1}:
Alex prend une tablette et prend son caddie muni d'une puce rfid.
Alex scanne sa carte client avec la tablette et son compte client s'affiche, il contient toutes ses listes de course.
\par
Alex retrouve la liste qu'il a précedement sauvegardée et la valide.
Alex commence ses courses, en cas de rupture de stock, les produits manquant sont en surbrillance sur la tablette et à chaque fois Alex choisit de selectionner la proposition alternative.
Alex suit le trajet proposé par la tablette, il met les produits dans son caddie muni d'un capteur rfid qui scanne chaque produit qu'on y met et coche l'élément correspondant sur la liste.En cas de non reconnaissance dans la liste, la tablette propose de l'associer avec un produit similaire et de simplement procéder a un ajout de produit dans la liste.
\par

Alex a parcouru toute sa liste, il se dirige vers la caisse. 
L'intégralité du caddie est scanné grace lecteur rfid installé sur le portique.
Sans avoir a sortir les éléments, Alex procède au paiement.
A la caisse, il est demandé à Alex de choisir son mode de paiement(carte bancaire, espèces ou chèque)
Alex choisit de payer par chèque. 
Alex récupère son ticket de caisse. 
Alex rend la accroche la tablette à une borne prévue pour cela.  
Alex retourne maintenant a sa voiture avec ses courses.

\item {\bf Variante 2}:
Alex prend une tablette et prend son caddie.
Alex scanne sa carte client avec la tablette et son compte client s'affiche, il contient toutes ses listes de course. 
\par
Alex retrouve la liste qu'il a précedement sauvegardée e la valide.
Alex commence ses courses, en cas de rupture de stock, les produits manquant sont en surbrillance sur la tablette et à chaque fois Alex choisit de selectionner la proposition alternative.
\par

Alex suit le trajet proposé par la tablette, il met les produits dans son caddie et coche l'élément correspondant sur la liste.En cas de non présence dans la liste , la tablette propose de l'ajouter manuellement.
\par

Alex a parcouru toute sa liste, il se dirige vers la caisse. 
Alex décharge la totalité de son caddie sur la caisse de Raissa la caissière.
Raissa lui dit bonjour, et commence à scanner les articles un par un.
Alex valide la liste sur la tablette, ceci met à jour les points sur sa carte de fidélité et enregistre la liste de course dans son espace client.
Raissa demande à Alex sa carte de fidélité.
Raissa scanne la carte et demande à Alex quel mode de paiement il compte utiliser.(Carte bancaire, espèces, ou chèque)
Alex règle la facture.
Raissa lui donne son ticket de caisse et lui souhaite une excellente journée.
Alex rend la tablette à Raissa qui s'occupera de la remettre sur la borne à tablette.
Alex retourne maintenant a sa voiture avec ses courses.
\stopitemize

