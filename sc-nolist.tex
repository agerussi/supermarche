\section{«À l'improviste»}

Raïssa sort du boulot assez tard et doit encore rapidement faire ses courses pour manger ce soir.
\par
À l'entrée du mini-futurmarket de son quartier, elle se saisit d'un petit panier sur roulettes et d'une markettab, puis commence par y consulter les bonnes occasions du moment.
Voyant que les cuisses de poulet sont en promo, elle se rend au rayon viande, prend un paquet de 2 cuisses de poulet, qu'elle scanne sur la tablette.
Elle sélectionne alors l'option «recettes associées» et consulte la liste des recettes possibles avec cet article (\in{figure}[POULET]).
\placefigure[][POULET]{Choix de recettes sur la markettab}{\externalfigure[CROQUIS/poulet_markettab.svg]}

Elle choisit «poulet basquaise» pour 2 personnes (elle est seule mais veut avoir des restes pour le lendemain), une recette bien notée par les consommateurs, et qui dont le temps de préparation est estimé à 45 minutes.
Elle transfert la liste des articles nécessaires dans un nouveau panier.
Après avoir déselectionné le riz et les divers condiments (elle sait en avoir encore à la maison), elle se laisse guider au plus court dans le magasin, recueille les articles restants (2 poivrons jaunes, des lardons, des tomates) puis se dirige vers la caisse.
Elle papotte un peu avec l'hotesse de caisse, Vanessa, qui est presque devenue son amie par le force du temps. 
Raïssa apprécie ce contact humain et ne comprend pas les gens qui ne jurent que par le cadbot.
\par
Arrivée à la maison, ayant un doute sur la façon de cuisiner son poulet, elle se connecte sur le site du futurmarket et imprime la recette culinaire du poulet basquaise.

