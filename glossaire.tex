\title{Glossaire métier}

\glossaire{Balise (de cadbot)}
Petit émetteur rattaché à un cadbot, que le client décroche et met dans sa poche afin de prendre possession du cadbot.
Un raccrochement de la balise donne au cadbot l'ordre de rejoindre son cadpark.
\par

\glossaire{Cadbot}
Un caddie robotisé, possédant de multiples fonctionnalités: suivi automatique du porteur de sa balise, déplacement autonôme vers son cadpark, suivi des articles déposés ou retirés grâce à des lecteurs de codes barres et une balance intégrée, et procédure de validation au niveau du portique de paiement.
\par

\glossaire{Cadpark}
Endroit où se rangent les cadbots, incluant un système de recharge sans contact de leur batterie.
\par

\glossaire{Caisse}
Désigne la traditionnelle caisse à hotesse, avec dépot, scan puis paiement des articles.
Ce type de caisse est progressivement remplacé par des portiques, mais il en reste toujours pour les clients qui le souhaitent, ou ceux qui n'utilisent pas un cadbot.
\par

\glossaire{Drive}
Désigne l'annexe d'un FuturMarket par laquelle on vient retirer ses courses en voiture, suite à une commande de panier en ligne depuis le site du futurmarket.
Les produits ont été retirés des rayons directement par un personnel du futurmarket et embalés, prêts à être retirés par le client.
\par

\glossaire{Euréka}
Borne interactive disposée un peu partout dans les FuturMarkets et consultable vocalement.
Permet de trouver rapidement un article ou d'avoir des renseignements concernant sa disponibilité ou un produit de remplacement.
Les bornes euréka sont essentiellement prévues pour les clients n'ayant pas pris une markettab à l'entrée dans le magasin, puisque cette dernière propose les mêmes fonctionnalités, et plus.
\par

\glossaire{Futurmarket}
Chaîne de supermarchés offrant un confort inégalé au client grâce à l'utilisation intensive des nouvelles technologies.
Par extension, instance d'un supermarché de la chaîne FuturMarket.
\par

\glossaire{Liste (de courses) ou Panier} 
Ensemble des articles que le client a prévu d'acheter dans le futurmarket.
Les listes des courses sous forme électronique sont préparées par le client à partir de l'application internet supermarket au moyen de différents systèmes d'assistance, et peuvent être sauvegardées et manipulées. 
Elles sont alors désignées sous le nom de {\em panier}.
Le client peut retrouver un panier sur les markettabs simplement par scan de sa carte client.
\par

\glossaire{Livraison}
Désigne l'option par laquelle le client choisit que ses courses lui soient livrées directement à domicile, à partir d'un panier constitué en ligne depuis le site du futurmarket.
Cette option induit un surcoût variable selon la taille du panier et la distance de livraison.
\par

\glossaire{Markettab}
Tablette électronique que l'on récupère à l'entrée dans le futurmarket et dépose à la sortie.
Elle offre diverses fonctionnalités telles le guidage dans le magasin à partir d'une liste de courses, la recherche d'un article, ou des conseils d'achats.
\par

\glossaire{Portique de paiement}
Zone de paiement pour les clients utilisant un cadbot.
Essentiellement, un cadbot ne peut quitter son portique qu'une fois le paiement effectué par le client.
\par

\glossaire{Recette (connectée)}
Dans le cadre d'un supermarket, une recette connectée une association de plusieurs articles, dans le but de confectionner un plat. 
Les recettes permettent de générer du contenu de panier automatiquement, par exemple en ligne, mais également de donner des idées d'achats à partir d'un simple scan d'article sur une markettab.
Une recette est en général accompagnée également d'un mode d'emploi, qui est en fait la recette de cuisine au sens traditionnel du terme.
\par 

\glossaire{Rfid (puce)}
Système de radio-identification permettant l'identification automatique et sans scan explicite d'un article.
Ce système est présent dans de nombreux articles, notamment les plus chers, afin de simplifier au maximum le travail du client (la puce est lue directement par le cadbot quant on y dépose l'article), mais aussi pour la sécurité supplémentaire par rapport à un scan classique.
\par

\glossaire{Scan}
Action consistant à faire lire le code barre d'un article, que ce soit par un cadbot, une markettab ou à une caisse.
\par
