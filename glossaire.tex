\chapter{Glossaire métier}

\glossaire{Article générique}
Un article générique est un ensemble d'articles concrets (\cad un produit précis, possédant une marque, une référence\dots) proposant les mêmes services.
C'est une classe d'équivalence sur les articles du FuturMarket pour la relation d'équivalence «remplaçable par».\crlf
Les articles génériques sont utilisés entre autres dans les recettes connectées, les bornes Euréka, le module de gestion de budget, pour réaliser des équivalences entre articles.

\glossaire{Balise (de cadbot)}
Petit émetteur rattaché à un cadbot, que le client décroche et met dans sa poche afin de prendre possession du cadbot.
Un raccrochage de la balise donne au cadbot l'ordre de rejoindre son cadpark.

\glossaire{Cadbot}
Un caddie robotisé, possédant de multiples fonctionnalités: suivi automatique du porteur de sa balise, déplacement autonome vers son cadpark, suivi des articles déposés ou retirés grâce à des lecteurs de codes barres et une balance intégrée, et procédure de validation au niveau du portique de paiement.

\glossaire{Cadpark}
Endroit où se rangent les cadbots, incluant un système de recharge sans contact de leur batterie.

\glossaire{Caisse}
Désigne la traditionnelle caisse à hôtesse, avec dépôt, scan puis paiement des articles.
Ce type de caisse est progressivement remplacé par des portiques, mais il en reste toujours pour les clients qui le souhaitent, ou ceux qui n'utilisent pas un cadbot.

\glossaire{Drive}
Désigne l'annexe d'un FuturMarket par laquelle on vient retirer ses courses en voiture, suite à une commande de panier en ligne depuis le site du futurmarket.
Les produits ont été retirés des rayons directement par un personnel du futurmarket et emballés, prêts à être retirés par le client.

\glossaire{Emplacement (article)}
Coordonnées permettant de désigner sans ambiguïté la place d'un article dans les rayons du magasin.
À noter qu'un même article peut posséder plusieurs emplacements.

\glossaire{Euréka}
Borne interactive disposée un peu partout dans les FuturMarkets et consultable vocalement.
Permet de trouver rapidement un article ou d'avoir des renseignements concernant sa disponibilité ou un produit de remplacement.
Les bornes euréka sont essentiellement prévues pour les clients n'ayant pas pris une markettab à l'entrée dans le magasin, puisque cette dernière propose les mêmes fonctionnalités, et plus.

\glossaire{Futurmarket}
Chaîne de supermarchés offrant un confort inégalé au client grâce à l'utilisation intensive des nouvelles technologies.
Par extension, instance d'un supermarché de la chaîne FuturMarket.

\glossaire{Liste (de courses) connectée} 
Ensemble d'articles que le client a prévu d'acheter dans le futurmarket.
Ces listes des courses sous forme électronique sont préparées par le client à partir de l'application internet supermarket au moyen de différents systèmes d'assistance, et peuvent être sauvegardées et manipulées. 
Le client peut retrouver une liste connectée sur les markettabs simplement par scan de sa carte client, et pourra alors l'inclure à son panier.

\glossaire{Livraison}
Désigne l'option par laquelle le client choisit que ses courses lui soient livrées directement à domicile, à partir d'un panier constitué en ligne depuis le site du futurmarket.
Cette option induit un sur-coût variable selon la taille du panier et la distance de livraison.

\glossaire{Markettab}
Tablette électronique que l'on récupère à l'entrée dans le futurmarket et dépose à la sortie.
Elle offre diverses fonctionnalités telles le guidage dans le magasin à partir d'un panier, la recherche d'un article, ou des conseils d'achats.
La Markettab peut se fixer sur le cadbot grâce au support prévu à cet effet.

\glossaire{Panier}
Ensemble des articles que le client veut acheter.
Le panier se gère sur la markettab, à partir d'une liste connectée préalablement préparée, et/ou à partir d'ajouts spontanés, utilisant ou non divers systèmes d'aide au choix.

\glossaire{Portique de paiement}
Zone de paiement pour les clients utilisant un cadbot.
Essentiellement, un cadbot ne peut quitter son portique qu'une fois le paiement effectué par le client. 
Un portique comporte des panneaux de détection, une caisse, des barrières de sortie du sas de paiement, et une lumière verte, orange ou rouge qui permet d'indiquer visuellement l'état dans lequel il se trouve.

\glossaire{Recette (connectée)}
Dans le cadre d'un supermarket, une recette connectée une association de plusieurs articles, dans le but de confectionner un plat. 
Les recettes permettent de générer du contenu de panier automatiquement, par exemple en ligne, mais également de donner des idées d'achats à partir d'un simple scan d'article sur une markettab.
Les ingrédients de recettes sont exprimés en termes d'articles génériques.
Une recette est en général accompagnée également d'un mode d'emploi, qui est en fait la recette de cuisine au sens traditionnel du terme.

\glossaire{Rfid (puce)}
Système de radio-identification permettant l'identification automatique et sans scan explicite d'un article.
Ce système est présent dans de nombreux articles, notamment les plus chers, afin de simplifier au maximum le travail du client (la puce est lue directement par le cadbot quant on y dépose l'article), mais aussi pour la sécurité supplémentaire par rapport à un scan classique.

\glossaire{Scan}
Action consistant à faire lire le code barre d'un article, que ce soit par un cadbot, une markettab ou à une caisse.

\glossaire{WoodStock}
Application de gestion des stocks à l'échelle d'un futurmarket.
Offre des fonctionnalités pour consulter l'état du stock, anticiper les ruptures, déclarer des commandes.
\par % ne pas supprimer
