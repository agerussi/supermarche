\section{«Pressé!»}

CongCong de chez lui se connecte sur le site du distributeur afin de préparer ses courses.
\placefigure[][lcon]{liste connectée avant login}{\externalfigure[CROQUIS/Liste_connectée_avt_login.svg][width=0.6\hsize]}
La, il retrouve ses achats précédents groupés par date d'achat, ainsi qu'un menu qui lui permet de créer sa liste selon les derniers achats, selon des recettes qu'il choisira, ou avec une restriction budgétaire (\in{figure}[lcon]). 
\placefigure[][lcon]{liste connectée apres login}{\externalfigure[CROQUIS/Liste_connectée_apres_login.svg][width=0.6\hsize]}
Aujourd'hui, Congcong est pressé et valide la proposition faite par le site, qui est de renouveler les achats de sa dernière liste, agrémentée de produits qu'il n'a pas acheté depuis longtemps (sopalin, savon, dentifrice et des filtres à café.)
Congcong valide puis imprime sa liste sans prendre la peine de la sauvegarder: c'est une liste imparfaite mais qui fera son office aujourd'hui.
\par
Arrivé au futurmarket, il passe chercher un cadbot, actionne le mode manuel et, connaissant le magasin par cœur, fonce à travers les rayons 
La plupart des produits possèdent un logo indiquant qu'une puce rfid est utilisée pour leur identification.
CongCong est donc libre de mettre ou retirer ces produits du cadbot sans devoir penser à les scanner à chaque opération.
Pour les autres produits, de plus en plus rares, le logo n'est pas présent et le scan du code barre est nécessaire.
\par
Lorsqu'il arrive à la caisse, le cadbot se dirige vers le premier portique de paiement libre.
CongCong satisfait de ses courses, procéde au paiement par carte Futurmarket de son caddie, choisissant l'option «crédit une semaine gratuit»: il est un peu limite sur son compte mais attend son salaire imminnement sous peu.
CongCong se dirige maintenant vers sa voiture pour décharger ses courses.
CongCong raccroche la balise au cadbot qui va automatiquement se ranger dans son cadpark de rattachement: moins de 20 minutes pour faire ses courses de la semaine, record battu.

\subsection{Variante: contrainte budgétaire}
CongCong de chez lui se connecte sur le site du distributeur afin de préparer ses courses.
Là, il retrouve ses achats précédents groupés par date d'achat.
Aujourd'hui, Congcong a une fin de mois difficile et sélectionne le menu budget, puis définit un montant à ne pas dépasser.
Le site lui propose une liste selon ses dernières listes d'achats, cependant les produits en plusieurs exemplaires et ceux pour lesquels il existe un produit moins cher (discount, marque distributeur, concurrent) apparaissent en surbrillance. 
Un bandeau de couleur rouge reste présent tant que le budget défini n'est pas atteint.
Cela n'empêche pas Congcong de valider sa liste malgré le léger dépassement de la restriction budgétaire.
CongCong se rend ultérieurement en magasin pour faire ses courses comme dans le scénario ci-dessus.
