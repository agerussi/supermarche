\section{«Les courses de la semaine»}

Congcong se connecte sur le site de son futurmarket.
Il sélectionne le menu 'liste connectée/aide au choix'.
Une liste générée par analyse de son historique d'achats lui est proposée.
Il ajoute ou supprime des éléments, notamment du coulis de framboise pour le dessert de dimanche.
Malheureusement le coulis de framboises est en rupture de stock.
Plusieurs produits similaires lui sont proposés, il sélectionne le coulis de fraise.
Il sauvegarde sa liste sous le nom "ma-liste-10/10/14".
\par
En entrant dans le magasin du distributeur, Congcong constate qu'il y a énormément de monde.
Il se saisit donc d'une markettab mais préfère prendre un caddie classique car il sait que le suivi automatique du cadbot peine parfois quand le trafic est dense.
Congcong scanne sa carte client avec la tablette et son compte client est affiché, ainsi que les listes classées dans l'ordre décroissant de validation.
CongCong choisit alors sa liste de courses "ma-liste-10/10/14" et le style de parcours: rapide.\par
Il passe ainsi récupérer le pain de mie, la viande, 2 packs d'eau, des curlys.
Un simple appui sur la markettab permet de lui indiqué de passer à l'article suivant.
\par
Arrivé au rayon petit déjeuner, il ne trouve pas ses céréales Kellog's extra chocolat.
Apparemment elles viennent d'être en rupture de stock.
En appuyant sur l'option \type{[variante]} la markettab lui propose d'autres produits similaires, mais Congcong n'étant pas convaincu, préfère carrément changer de produit.
\par
Une fois la liste de courses parcourue, Congcong se dirige vers la caisse pour régler ses courses. 
Il décharge son caddie sur le tapis roulant de la caisse de Manon, qui commence à scanner chaque article.
Cette opération prend de moins en moins de temps depuis que certains produits possèdent des puces rfid, et qui ne demandent plus que des manipulations très limitées pour être détectés.
\par
Quand Manon a fini de scanner les articles elle demande à Congcong sa carte de fidélité et la scanne pour la tenir à jour.
Manon indique à CongCong le montant à payer et lui demande quel mode de paiement il utilise.
CongCong choisit de payer par carte bancaire. 
Manon le remercie et lui souhaite de passer une bonne journée.
Congcong dépose sa markettab en sortie de magasin et retourne à sa voiture avec ses courses.

