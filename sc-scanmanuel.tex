\section{liste connectée, tablette, scan manuel}

CongCong se connecte sur le site du distributeur.
Il sélectionne le menu 'aide au choix'.
La liste générée à partir de ses derniers achats lui est proposée.
Il ajoute ou supprime des éléments.
En cas de rupture, CongCong a le choix parmi plusieurs propositions de produits similaires.
En entrant dans le magasin du distributeur, CongCong prend un caddie et une tablette.
CongCong scanne sa carte client avec la tablette et son compte client est affiché, ainsi que les listes classées dans l'ordre décroissant de validation.
CongCong choisit alors sur la tablette, la liste de courses du moment ainsi que le style de parcours ( rapide, decouverte, gourmand)
en cas de rupture de stock, les produits manquants sont en surbrilliance, CongCong choisit selectionner la proposition alternative a chaque fois.
Suivant le trajet proposer par la tablette, CongCong mets les produits dans son caddie.
\startitemize
\item {\bf Variante 1}: 
Une fois, la liste de courses parcourue, Congcong se dirige vers la caisse pour regler ses courses. 
Il décharge son caddie sur le tapis roulant de la caisse de Manon.
Manon lui dit bonjour et commence à scanner chaque article.
Quand Manon fini de scanner les articles elle demande à CongCong sa carte de fidélité et la scanne.(ou la liste est validé sur la tablette et cela met la carte de fidélité à jour).
Manon indique à CongCong le montant à payer et lui demande quel mode de paiement il utilise(carte bancaire, chèque ou escpèces).
CongCong choisit de payer par carte bancaire. 
CongCong règle la facture
Manon lui remercie et lui souhaite de passer une bonne journée.
Congcong retourne maintenant a sa voiture avec ses courses.
\item {\bf Variante 2}: 
CongCong choisit alors sur la tablette, la liste de courses du moment ainsi que le style de parcours ( rapide, decouverte, gourmand)
En cas de rupture de stock, les produits manquants sont en surbrilliance, CongCong choisit selectionner la proposition alternative a chaaque fois.
Suivant le trajet proposer par la tablette, CongCong mets les produits dans son caddie et le capteur rfid coche les élements passé dans le caddie. En cas de non reconnaissance dans la liste, la tablette propose de l'associer avec un produit similaire et de simplement procéder a un ajout de produit dans la liste.
Une fois, la liste de courses parcourue, Congcong se dirige vers le terminal de paiement. L'intégralité du caddie est scanné grace lecteur rfid installé sur le portique.
Sans avoir a sortir les éléments, CongCong procède au paiement.
Congcong retourne maintenant a sa voiture avec ses courses.
\stopitemize
