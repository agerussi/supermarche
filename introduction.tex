\chapter{Introduction}

Modéliser un supermarché, futuriste ou non, est une vaste tâche.
Voulant privilégier l'aspect futuriste, nous nous sommes, durant cette première phase d'analyse, concentrés essentiellement sur des fonctionnalités nouvelles à destination du client.
C'est en effet dans ce domaine où, en temps que clients nous-mêmes, nous avons le plus d'expertise et donc, espérons-le, de pertinence.\par
Nous n'avons proposé de scénarios que dans la mesure où ils faisaient apparaître des aspects novateurs.
Pour cela, nous avons envisagé les quatre grandes étapes incontournables d'un client de supermarché:
\startitemize[n]
\item Le choix des articles: nous proposons des fonctionnalités nouvelles autour de la notion de liste de course connectée, et également de recette connectée;
\item La récupération des articles en magasin ou en dehors: nous proposons ici plusieurs éléments facilitant leur localisation, leur transport, et le parcours effectué;
\item le passage en caisse: dans ce domaine nous espérons des gains de temps très importants grâce à un système de caddie intelligent, assisté éventuellement par la technologie rfid;
\item le paiement proprement dit: c'est un domaine difficile à révolutionner car déjà très développé actuellement.
\stopitemize
\blank
Dans le domaine «miroir» des aspects du système à l'adresse du personnel des supermarchés, nous nous sommes assez vite heurtés à notre méconnaissance du milieu.
C'est pourquoi nous nous sommes contentés d'effleurer ce vaste sujet.

