\section{classique eureka cadbot cb rfid» (WIP)}

\par

\startitemize
\item Congong arrive dans le supermarché avec sa liste de course à la main. il a prévu d'acheter du 5 packs de lait, 2 kg de viande, 16 yaourts a la fraise , 2 boites de biscuits petit lu, un sachet de carambar, une feuille de chaine, 2 kg d'orange.
\item Il passe par le cadpark le plus proche et récupère un cadbot, décroche sa balise et la met dans sa poche.[CROQUIS 1]
\item Il se promène dans les rayons, suivi automatiquement par le cadbot.
\item Il scanne chaque article grâce à la douchette intégrée au cadbot.[CROQUIS 2]
\item Celui-ci détecte le dépot d'un article, et en vérifie l'authenticité grâce à son système 
de balance intégrée (le poids de l'article est codé dans le code barre).
\item Plus loin, Congong cherche désespérément les petits suisses yoplait.
\item Il se dirige en tête de gondole vers la borne euréka la plus proche et prononce les mots "petits suisses" tout en appuyant sur le bouton prévu à cet effet.[CROQUIS 3]
\item Sur l'écran s'affiche alors la disponibilité de tous les produits assimilés à des petits suisses.[ECRAN 1]
\item Après avoir choisi un élément, celui est localisé sur le plan schématique du magasin qui apparait.
\item Ayant trouvé ses petits suisses, il décide de reposer les yaourts à la fraise dans le rayon.
\item Il les retire de son caddie et les scanne à nouveau.
\item Il se dirige vers le terminal de paiement, passe le portique et le prix de l'ensemble de ses courses s'affiche sur la borne dès que son cadbot vient se garer au milieu du portique, où il restera bloqué en attendant le paiement.[CROQUIS 4]
\item Congong paie avec sa carte banquaire et va à sa voiture où il décharge le cadbot.
\item Il raccroche alors la balise au cadbot, qui va automatiquement se ranger dans son cadpark de rattachement.
\stopitemize

\par


\par

\subsection{Variantes}

Variante1: scanne du caddie par RFID.

\startitemize
\item CongCong ne scanne plus les articles avant de les déposer dans le caddie, mais les dépose directement.
\item il est donc libre de mettre et enlever les produits sans devoir penser a les scanner a chaque operation (ajout/suppression)
\item Lorsqu'il arrive a la caisse, le cadbot se dirige vers un emplacement de detection des tags rfid des produits.
\item Sans rien sortir de son caddie, chaque produit est identifié et CongCong prend son ticket de caisse.
\item CongCong satisfait de ses courses, procéde au paiement par CB de son caddie.
\item CongCong se dirige maintenant vers sa voiture pour decharger ses courses.
\item CongCong raccroche la balise au cadbot qui va automatiquement se ranger dans son cadpark de rattachement.
\stopitemize

\par

\par
Variante2: Contrainte sur la liste connectée (WIP)



\par


\par
\subsection{Glossaire}
 -cadbot: chariot motorisé qui suit le porteur du tag RFID grace a des capteurs de proximité
 -cadpark: lieu ou les cadbot se rendent de manière autonome pour se recharger.
 -borne eureka: borne tactile/vocale d'interrogation de disponibilité et de localisation de produit.
 La borne peut egalement proposé des produits de substitution en cas de rupture de stock.
  \par
