\section{«Vite et bien»}

Congcong arrive dans le supermarché avec la liste de courses griffonnée à la va-vite sur un bout de papier: voilà ce qui arrive quand on l'internet maison est en panne, et pas encore de smartphone.
Il doit acheter 5 packs de lait, du cacao, un poulet fermier, une baguette, un pot de miel, un kilo de sucre, 2 paquets de biscuits au chocolat, 2 kilos de pommes de terre, de la creme fraiche, des bananes, des petits suisses yoplait.
\par
Il passe par le cadpark le plus proche et récupère un cadbot, décroche sa balise et la met dans sa poche (\in{figure}[BOTSUIVI]).
Il se promène dans les rayons, suivi automatiquement par le cadbot.
Il scanne chaque article grâce à la douchette intégrée au cadbot (\in{figure}[BOTSCAN]).
Celui-ci détecte le dépot d'un article, et en vérifie l'authenticité grâce à son système de balance intégrée (le poids de l'article est codé dans le code barre).
\par
Plus loin, Congong cherche désespérément les petits suisses yoplait.
Il se dirige en tête de gondole vers la borne euréka (\in{figure}[BOEUR]) la plus proche et prononce les mots «petits suisses» tout en appuyant sur le bouton prévu à cet effet.
Sur l'écran s'affiche alors la disponibilité de tous les produits assimilés à des petits suisses.
Après avoir choisi un élément, celui est localisé sur le plan schématique du magasin qui apparait.
Ayant trouvé ses petits suisses yoplait, qui ont été déplacés au rayon yaourths, il décide de reposer les yaourths à la fraise au passage.
\placefigure[][BOEUR]{Borne Eureka}{\externalfigure[CROQUIS/borneEureka.jpg][width=\hsize]}
Il les retire donc de son caddie et les scanne à nouveau.
\par
Congcong termine ses courses en ajoutant dans le panier les produits de son choix.
Il se dirige vers le terminal de paiement, passe le portique et le prix de l'ensemble de ses courses s'affiche sur la borne dès que son cadbot vient se garer au milieu du portique, où il restera bloqué en attendant le paiement (\in{figure}[BOTPORTIQUE]).
Congcong paie avec sa carte banquaire et va à sa voiture où il décharge le cadbot.
Il raccroche alors la balise au cadbot, qui va automatiquement se ranger dans son cadpark de rattachement.
\placefigure[][BOTSUIVI]{Le cadbot suit le porteur de sa balise}{\externalfigure[CROQUIS/cadbot-suivi.svg][width=0.6\hsize]}
\placefigure[][BOTSCAN]{Dépot des articles dans un cadbot}{\externalfigure[CROQUIS/cadbot-scan.svg][width=0.5\hsize]}
\placefigure[][BOTPORTIQUE]{Portique de paiement}{\externalfigure[CROQUIS/cadbot-portique.svg][width=0.6\hsize]}
