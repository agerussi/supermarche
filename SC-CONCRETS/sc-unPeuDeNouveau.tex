\section{«Un peu de nouveau!»}
Alex se connecte sur le site du futurmarket et décide qu'il est temps de composer une liste des courses un peu originale pour changer de la routine habituelle.
En effet, à force d'utiliser toujours plus ou moins ses vieilles listes de courses, il a constaté qu'il mangeait toujours un petit peu la même chose pendant l'année !\par
Il sélectionne donc \type{[liste connectée/recettes]}.
Alex sélectionne autant de recettes qu'il prévoit de repas ainsi que les options associés (entre autres: végétarien, kasher, halal, nombre de personnes, temps de préparation, \dots).
Il lance la composition du panier.
Parcourant rapidement la liste des yeux, Alex ajoute à la main des produits ménagers dont il a besoin, et en supprime deux ou trois qu'il a encore dans son frigo, puis la sauvegarde.
Alex choisit pour finir d'imprimer deux nouvelles recettes culinaires associées aux recettes connectées qu'il a choisi: «bœuf bourguignon à la provençale» et «lapin chasseur», afin que sa femme ait une idée de départ concernant la préparation de ces plats un peu originaux.
\par
Il va alors vers la cuisine et annonce à sa femme qu'il vient de la débarrasser de sa corvée de la semaine: avoir des idées de repas !
Il a oublié un détail: demain samedi, jour où il fait d'ordinaire les courses, il est en vadrouille pour son club de sport !
Il se reconnecte alors au site du futurmarket, remet la main sur sa liste de course précédente, et choisit de se faire livrer pour demain soir: malgré la surfacturation de 20.15€ qui lui est annoncée, il valide.
