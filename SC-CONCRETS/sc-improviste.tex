\section{«À l'improviste»}

Raïssa sort du boulot assez tard et doit encore rapidement faire ses courses pour manger ce soir.
\par
À l'entrée du mini-futurmarket de son quartier, elle se saisit d'un petit panier sur roulettes et d'une markettab, puis commence par y consulter les bonnes occasions du moment.
Voyant que les cuisses de poulet sont en promo, elle se rend au rayon viande, prend un paquet de 2 cuisses de poulet, qu'elle scanne sur la tablette.
Elle sélectionne alors l'option «recettes associées» et consulte la liste des recettes possibles avec cet article (\in{figure}[IMPROPOULET]).
\placefigure[][IMPROPOULET]{Choix de recettes sur la markettab}{\externalfigure[CROQUIS/poulet_markettab.svg]}
Elle choisit «poulet basquaise» pour 2 personnes, une recette bien notée par les consommateurs, et dont le temps de préparation est estimé à 45 minutes.
Elle choisit d'appliquer cette recette à son panier. 
Dans le menu de mise à l'échelle (\in{figure}[IMPROSCALE]), elle sélectionne «2 personnes»: elle est seule mais veut avoir des restes pour le lendemain.
\placefigure[][IMPROSCALE]{Mise à l'échelle d'une recette sur la markettab}{\externalfigure[CROQUIS/miseechelle_markettab.svg][height=5truecm]}
La liste des articles nécessaires apparaît, avec les quantités adaptées pour 2 personnes.
Elle désélectionne le riz et les divers condiments (elle sait en avoir encore à la maison) et valide.
Son panier est maintenant rempli des articles nécessaires.
Elle sélectionne l'application de guidage, puis se laisse guider au plus court dans le magasin, recueille les articles restants (2 poivrons jaunes, des lardons, des tomates): il lui suffit de sélectionner l'article suivant pour qu'il apparaisse sur le plan affiché sur la markettab.
Enfin, elle se dirige vers les caisses traditionnelles.
Elle papote un peu avec l'hôtesse de caisse, Vanessa, qui est presque devenue son amie par le force du temps. 
Raïssa apprécie ce contact humain et ne comprend pas les gens qui ne jurent que par le cadbot, même s'il est vrai qu'il fait gagner du temps.
