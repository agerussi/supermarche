\section{Journée d'une hôtesse de caisse}

Mariline arrive au supermarché à 08h45. 
Elle commence son poste du matin à 09h00. 
Elle passe au vestiaire pour se changer. 
ELle prend son badge RFID pour pointer son heure d'arrivée (RFID faible distance), la tablette fixée au mur enregistre l'heure et lui indique la caisse où il est prévu qu'elle travaille (n°58).
Elle passe prendre le tiroir caisse n°58.
Arrivée à la caisse n°58, Mariline s'installe et pose son tag RFID pour déverouiller le logiciel de la caisse. 
ELle peut maintenant insérer sa caisse monetique dans l'appareil: elle est prete pour recevoir les clients. 
\par
Le premier client arrive, pose ses courses sur le tapis roulant. 
Mariline prend les produits un par un pour les scanner: il y a encore des gens qui préfèrent passer par elle plutôt que de s'«encombrer d'un cadbot», comme ils disent.
Au moins, ça lui permet de garder son boulot !
Passent alors, la farine, le sucre, la levure, la confiture, du sirop d'erable, et du cidre. 
Mariline se dit qu'elle aussi ferait bien des crèpes ce soir. 
\par
Lorsque les produits sont scannés, elle demande au client s'il a sa carte client. 
Ce dernier lui la tend, elle peut ainsi enregistrer le panier client sur le compte du client.
Toutes les heures, une collègue vient prélever les billets par lots de 500 euros. 
La collegue a un badge special qui valide le retrait des lots de 500 euros dans le logiciel de caisse. 
Après avoir traité 94 clients ce jour, elle quitte son poste sans oublier de repartir avec le tiroir caisse et de verrouiller le logiciel de caisse. 
Mariline depose alors le tiroir caisse dans le local securisé, se change et repart chez elle en badgeant avant de partir.
