\section{«Mise en rayon»}
Mickaella vient d'être appelée par son manager: des clients ne trouvent pas les allumettes: apparemment il doit en rester car les bornes Euréka signalent que le produit est encore en stock, mais l'emplacement indiqué n'est pas le bon.
Mickaella se souvient que lors du dernier déménagement, les allumettes ont été déplacées au rayon produits ménagers.
Elle se rend donc là-bas et y trouve les allumettes.
Elle scanne un paquet avec sa markettab (son modèle possède des fonctions activées spécialement pour le personnel), cherche les emplacements du produit: il n'y en a qu'un, et ce n'est effectivement pas le bon (voir \in{figure}[FEUDOR]).
%\placefigure[][FEUDOR]{La place des allumettes, sous markettab}{\externalfigure[CROQUIS/feudor.svg]}
\placefigure[][FEUDOR]{La place des allumettes, sous markettab}{\externalfigure[CROQUIS/marketTab_allumettes.png][height=0.3\vsize]}
Elle annule alors l'emplacement erroné, puis ajoute un nouvel emplacement en scannant le marqueur de rayon correspondant: en effet les rayons sont décomposés en zones d'un mètre de long, marquées par de discrets codes barres.
Les allumettes sont remises à leur place.
