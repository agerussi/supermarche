% exemple de rédaction des scénarii associés à un CU
\subsection{«LibérerCadbot»}
\TODO\ ce CU n'est pas intégrable tel quel: 
* devrait de focaliser sur l'interaction avec le client, le retour automatique est HS
* on ne sait pas comment l'anomalie est transmise
* que fait le cadbot quand il est presque déchargé ?
* que se passe-t-il si le cadbot n'est pas vide et qu'on raccroche la balise ?
* etc.

\startCU
\nom LibérerCadbot
\but  ce cas montre la suite de pas permettant à un client de libérer le cadbot une fois ses courses terminées.
\acteur Client
\precondition Le cadBot est vide
\auteur Raissa Mbabazi Simbi
\date 7/11/14

\nominal % scénario nominal
\startnominal
\etape[LC:SA1] Le client remet la balise dans son port sur le cadbot. 
\etape Le Cadbot détécte la balise.
\etape[LC:RETOUR] Le cadbot suit le chemin indiqué par son système pour regagner le parking des cadbot le plus proche.
\etape Le cadbot arrive à son parking.
\etape Le cadbot s'accroche sur une place libre de la grille 
\etape {\it fin du cas d'utilisation}
\stopnominal
\postcondition Le cadbot est dans le parking.

% les scénarii alternatifs
\alternatifs
\startalternatif[LC:SA1] % ensemble des scénarii débutant à SA1
\startcondition[Cadbot ne détecte pas la balise dans son port] 
  \etape Le client active le retour au parking manuellement.
	\etape retour à l'étape \in[LC:RETOUR]

\stopcondition
\postcondition L'anomalie concernant ce cadbot sera transmise à la maintenance.
\stopalternatif



% les scénarii d'exceptions
\exception
le cadbot est déchargé.
\stopCU
