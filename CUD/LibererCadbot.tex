\subsection{«LibérerCadbot»}
\startCU
\nom LibérerCadbot
\but ce cas montre la suite de pas permettant à un client de libérer le cadbot une fois ses courses terminées.
\acteur Client
\precondition Le cadBot est vide
\auteur Raissa Mbabazi Simbi
\date 7/11/14
\nominal % scénario nominal
\startnominal
\etape[LC:SA1] Le client remet la balise dans son port sur le cadbot.
\etape[LC:SA2] Le Cadbot détécte la balise.
\etape[LC:RETOUR] Le cadbot suit le chemin indiqué par son système pour regagner le parking des cadbot le plus proche.
\etape {\it fin du cas d'utilisation}
\stopnominal
\postcondition Le cadbot s'éloigne du client.
% les scénarii alternatifs
\alternatifs
\startalternatif[LC:SA1] % ensemble des scénario débutant à SA1
\startcondition[Cadbot ne détecte pas la balise dans son port]

\etape Le client indique que le cadbot est en panne par validation du cas d'urgence "En panne".
\etape Le client active le retour au parking manuellement.
\etape retour à l'étape \in[LC:RETOUR]
\stopcondition
\postcondition Par l'appui sur le bouton En Panne, L'annomalie de ce cadbot est transmis à la maintenance.
\stopalternatif
\startalternatif[LC:SA1] % ensemble des scénario débutant à SA1
\startcondition[On raccroche la balise alors que le cadbot n'est pas vide]
\etape Le cadbot émet des signales sonores consécutifs jusqu'à ce que la balise soit enlevée.
\stopcondition
\postcondition Le client a enlevé la balise.
\stopalternatif
\startalternatif[LC:SA2] % ensemble des scénario débutant à SA1
\startcondition[Le cadbot est presque déchargé]
\etape Le cadbot indique par deux longs bips  q'il est presque déchargé
\etape Le client valide le cas d'urgence corréspondant (batterie faible)
\stopcondition
\postcondition Par La validation du cas d'urgence batterie faible, Le cadbot va être récupéré sur le parking.
\stopalternatif

% les scénarii d'exceptions
\exception
le cadbot est déchargé.
\stopCU
