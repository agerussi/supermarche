\subsection{«UtiliserGuidageMarkettab»}
\TODO\ Ce scénario doit être revu, la markettab n'a pas (plus) de GPS, donc elle ne peut repérer le client, ni le guider... juste proposer un parcours sur un plan...

\startCU
\nom UtiliserGuidageMarkettab
\but Ce cas montre la suite de pas permettant à un client d'utiliser l'option de guidage dans le FuturMarket par la markettab
\acteur Client
\precondition Le client a sélectionné sa liste de course sur la marketTab, après avoir scanné sa carte Client. Pour commencer, le produit le plus proche du client est sélectionné.
\auteur Raissa Mbabazi Simbi, François Dubiez
\date 7/11/14, Rev 09/11/2014.
\nominal % scénario nominal
\startnominal
\etape[UGM:SA1] La markettab ouvre un plan du magasin, positionne le client sur celui ci et indique le chemin le plus court pour aller au produit de la liste, le plus proche.
\etape[UGM:RETOUR2] Le client en suivant les indications sur la carte, arrive à l'emplacement du produit cherché, prend le produit et sélectionne le suivant sur la marketTab.
\etape La marketTab calcule la suite de l'itinéraire.
\etape {\it fin du cas d'utilisation}
\stopnominal
\postcondition La marketTab affiche l'itinéraire pour aller au produit sélectionné.

% les scénarii alternatifs
\alternatifs
\startalternatif[UGM:SA1]
\startcondition[Changement du type de parcours.]
\etape Le client clique sur parcours découverte.
\etape Le système recalcule un itinéraire pour faire découvrir de nouveaux produits dans le domaine de ceux listés parmi la liste de courses. Le système ajoute ces suggestions à la liste de courses avec un maximum de 2 articles à découvrir par article présents dans la liste.
\etape retour à l'étape \in[UGM:RETOUR2]
\stopcondition
\postcondition La marketTab a complété la liste de courses avec des articles à découvrir.
\stopalternatif
\startalternatif[UGM:SA1]
\startcondition[ Le système informe la Markettab qu'un article est en rupture de stock ]
\etape La MarketTab informe le client que l'article est en rupture de stock et propose une suggestion de remplacement en indiquant une zone du magasin comportant ce type de produit.
\etape retour à l'étape \in[UGM:RETOUR2]
\stopcondition
\postcondition La marketTab a proposé un produit de remplacement.
\stopalternatif
\startalternatif[UGM:RETOUR2]
\startcondition[ Le client décide de sortir du chemin proposé par la Markettab]
\etape La Markettab calcule un nouveau chemin qui correspond à l'emplacement actuel du client. 
\etape La markettab ouvre une carte, positionne le client sur la carte et indique le chemin à suivre sur la carte.
\etape retour à l'étape \in[UGM:RETOUR2]
\stopcondition
\postcondition La carte du chemin à suivre est affichée sur la Markettab.
\stopalternatif
\startalternatif[UGM:SA1]
\startcondition[ La marketTab ne positionne plus le client ]
\etape La MarketTab demande au client de scanner un objet à proximité pour l'aider à se positionner. 
\etape Le Client scanne un produit du rayon.
\etape La marketTab recherche ce produit dans le plan et définit sa position.
\etape Retour à l'étape \in[UGM:SA1]
\stopcondition
\postcondition La carte du chemin à suivre est affichée sur la Markettab.
\stopalternatif

% les scénarii d'exceptions
\exception
\startalternatif[UGM:SA1]
\startcondition[article équivalent indisponible]
  \etape La marketTab informe le client qu'aucun article équivalent n'est disponible et propose de passer au suivant.
  \etape Le client valide le passage a l'article suivant.
  \etape retour à l'étape \in[UGM:RETOUR2]
\stopcondition
\postcondition La Markettab connaît le produit suivant à chercher.
\stopalternatif
\stopCU
