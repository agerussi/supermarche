\subsection{«UtiliserGuidageMarkettab»}
\TODO\ Ce scénario doit être revu entièrement, il ne correspond pas aux descriptions dans les scénarios concrets ou dans les CU (pour rappel: pour raison de difficultés technologiques, le guidage se "contente" d'afficher une carte et de positionner les articles dessus, voir par exemple "à l'improviste")

\startCU
\nom UtiliserGuidageMarkettab
\but Ce cas montre la suite de pas permettant à un client d'utiliser l'option de guidage dans le FuturMarket par la markettab
\acteur Client, le vigile
\precondition Le client prend une markettab.
\auteur Raissa Mbabazi Simbi
\date 7/11/14
\nominal % scénario nominal
\startnominal
\etape[UGM:RETOUR1] Le client indique la liste des produits qu'il souhaite acheter, soit par une liste connectée, soit par une composition de sa liste sur place.
\etape[UGM:SE1] La markettab ouvre une carte, positionne le client sur la carte et indique le chemin à suivre sur la carte.
\etape[UGM:RETOUR2] Le client se déplace dans le FuturMarket en suivant les indications sur la carte.
\etape {\it fin du cas d'utilisation}
\stopnominal
\postcondition La carte du chemin à suivre est affichée sur la Markettab.
% les scénarii alternatifs
\alternatifs
\startalternatif[UGM:RETOUR1]
\startcondition[La Markettab ne fonctionne pas]
\etape Le client remet la Markettab au vigile qui s'occupe également des tablettes.
\etape retour à l'étape \in[UGM:RETOUR1]
\stopcondition
\postcondition le vigile indique au système que la marketTab est déféctueuse.
\stopalternatif
\startalternatif[UGM:SE1]
\startcondition[ La Markettab ne retrouve pas dans le FuturMarket certains articles sur la liste de produits indiquée par le client]
\etape La Markettab recherche le produit générique correspondant et indique le chemin dans le FuturMarket. La MarketTab informe le client que cet info est moins précise que les précédentes.
\etape retour à l'étape \in[UGM:RETOUR2]
\stopcondition
\postcondition La carte du chemin à suivre est affichée sur la Markettab
\stopalternatif
\startalternatif[UGM:SE1]
\startcondition[ Le client décide de passer sortir du chemin proposé par la Markettab]
\etape La Markettab ferme la carte, et calcule un nouveau chemin qui correspond à l'emplacement actuel du client. 
\etape La markettab ouvre une carte, positionne le client sur la carte et indique le chemin à suivre sur la carte.
\etape retour à l'étape \in[UGM:RETOUR2]
\stopcondition
\postcondition La carte du chemin à suivre est affichée sur la Markettab.
\stopalternatif
% les scénarii d'exceptions
\exception
Aucun article sur la liste indiquée par le client n'est retrouvé dans le FuturMarket
\stopCU
