% exemple de rédaction des scénarii associés à un CU
\subsection{«UtiliserGuidageMarkettab»}
\TODO\ Pas intégrable en l'état:
* Il manque beaucoup de détails (par exemple étape 4: comment ???) 
* incohérences: par exemple en pré-condition: le client indique la liste des produits, et en étape 1, le client saisit une markettab...

\startCU
\nom UtiliserGuidageMarkettab
\but  ce cas montre la suite de pas permettant à un client d'utiliser l'option de guidage dans le FuturMarket par la markettab
\acteur Client
\precondition Le client indique la liste des produits qu'il souhaite acheter dans le FuturMarket.

\auteur Raissa Mbabazi Simbi
\date 7/11/14

\nominal % scénario nominal
\startnominal
\etape[UGM:RETOUR1] Le client saisit une markettab. 
\etape Le client indique la liste des produits qu'il souhaite acheter, soit par une liste connectée, soit par une composition de sa liste sur place.
\etape[UGM:SE1]  La markettab ouvre une carte, positionne le client sur la carte et indique le chemin à suivre sur la carte.
\etape[UGM:RETOUR2] Le client suit le chemin.
\etape {\it fin du cas d'utilisation}
\stopnominal
\postcondition La carte du chemin à suivre est affichée sur la Markettab

% les scénarii alternatifs
\alternatifs
\startalternatif[UGM:RETOUR1]
\startcondition[La Markettab ne fonctionne pas] 
  \etape Le client remet la Markettab sur son support
  \etape retour à l'étape \in[UGM:RETOUR1]
\stopcondition
\postcondition La carte du chemin à suivre est affichée sur la Markettab
\stopalternatif

\startalternatif[UGM:SE1]
\startcondition[ La Markettab ne retrouve pas dans le FuturMarket certains articles sur la liste de produits indiquée par le client]
   \etape  La Markettab indique le chemin corréspondant aux articles trouvés dans le FuturMarket.
  \etape retour à l'étape \in[UGM:RETOUR2]
\stopcondition
\postcondition La carte du chemin à suivre est affichée sur la Markettab
\stopalternatif


% les scénarii d'exceptions
\exception
Aucun articles sur la liste indiquée par le client n'est retrouvé dans le FuturMarket
\stopCU
