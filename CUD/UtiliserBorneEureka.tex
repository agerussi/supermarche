\subsection{«UtiliserBorneEuréka»}

\startCU
\nom UtiliserBorneEuréka
\but le client interroge une borne Euréka pour localiser un produit ou s’informer de sa disponibilité
\acteur un client
\precondition le client est dans le magasin face à une borne Euréka
\declenchement le client appuie sur le bouton d'interrogation
\auteur Alexandre G
\date 30/10/14

\nominal % scénario nominal
\startnominal
\etape[UBE:DEBUT] le client prononce le nom de l'article devant le microphone de la borne, puis relâche le bouton
\etape[UBE:INCONNU] le système affiche les mots prononcés, la liste des produits associés à ces mots ainsi que des renseignements (disponibilité, quantité en stock, prix\dots) 
\etape[UBE:LISTEVIDE] le client sélectionne un produit dans la liste
\etape le système affiche l'emplacement du produit et la position du client sur un plan du magasin
\etape[UBE:AUTRE] le système attend une minute avant d'afficher une page d'accueil
\etape {\it fin du cas d'utilisation}
\stopnominal
\postcondition le système affiche une page d'accueil

% les scénarii alternatifs
\alternatifs 

\startalternatif[UBE:INCONNU]
\startcondition[les mots prononcés ne sont pas reconnus]
  \etape le système affiche un message d'information
  \etape retour à l'étape \in[UBE:AUTRE]
\stopcondition
\stopalternatif

\startalternatif[UBE:LISTEVIDE]
\startcondition[la liste d'articles est vide]
\etape retour à l'étape \in[UBE:AUTRE]
\stopcondition
\stopalternatif

\startalternatif[UBE:AUTRE] 
\startcondition[le client veut sélectionner un autre article de la liste] 
  \etape le client revient à la liste 
  \etape retour à l'étape \in[UBE:LISTEVIDE]
\stopcondition
\startcondition[le client appuie sur le bouton d'interrogation]
  \etape retour à l'étape \in[UBE:DEBUT]
\stopcondition
\stopalternatif

\stopCU
