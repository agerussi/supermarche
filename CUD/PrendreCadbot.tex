% exemple de rédaction des scénarii associés à un CU
\subsection{Cas d'utilisation: «PrendreCadbot» }

\startCU
\nom Prendre Cadbot
\but  ce cas montre la suite de pas permettant à un client de prendre possession d'un cadbot pendant toute la durée de ses courses.
\acteur Client
\precondition Le client veut faire ses courses dans le FuturMarket.

\auteur Raissa Mbabazi Simbi
\date 7/11/14

\nominal % scénario nominal
\startnominal
\etape[SA1] Le client arrive au parking des cadbot. 
\etape[RETOUR] 	Le client prend un cadbot de la grille.
\etape[SE1] Le client prend la balise du cadbot.
\etape Le client met en marche le cadbot.
\etape Le cadbot suit le client tout au long du FuturMarket
\etape {\it fin du cas d'utilisation}
\stopnominal
\postcondition Le cadbot n'est plus dans le parking.

% les scénarii alternatifs
\alternatifs
\startalternatif[SA1] % ensemble des scénario débutant à SA1
\startcondition[Cadbot ne fonctionne pas] 
  \etape Le client remet le cadbot sur la grille.
  \etape retour à l'étape \in[RETOUR]
\stopcondition
\postcondition L'anomalie concernant ce cadbot sera transmise à la maintenance.
\stopalternatif

\startalternatif[SE1]
\startcondition[ Le cadbot ne detecte pas sa balise]
   \etape  Le client remet le cadbot sur la grille.
  \etape retour à l'étape \in[RETOUR]
\stopcondition
\postcondition L'anomalie concernant cet objet sera transmise à la maintenance.
\stopalternatif


% les scénarii d'exceptions
\exception
Il n'y a plus de cadbot dans la parking
\stopCU
