\subsection{«PrendreCadbot»}
\startCU
\nom Prendre Cadbot
\but ce cas montre la suite de pas permettant à un client de prendre possession d'un cadbot pendant toute la durée de ses courses.
\acteur Client
\precondition Le client veut faire ses courses dans le FuturMarket.
\auteur Raissa Mbabazi Simbi
\date 7/11/14
\nominal % scénario nominal
\startnominal
\etape[PC:SA1] Le client arrive au parking des cadbot.
\etape[PC:RETOUR] Le client prend un cadbot de la grille.
\etape[PC:SE1] Le client prend la balise du cadbot.
\etape Le client met en marche le cadbot en appuyant sur le bouton Marche/Arrêt.
\etape Le cadbot suit le client tout au long du FuturMarket
\etape {\it fin du cas d'utilisation}
\stopnominal
\postcondition Le cadbot n'est plus dans le parking.
% les scénarii alternatifs
\alternatifs
\startalternatif[PC:SA1] % ensemble des scénario débutant à SA1
\startcondition[Cadbot ne fonctionne pas]
\etape Le client indique que le cadbot est en panne par validation du cas d'urgence "En panne".
\etape Le client remet la balise sur le cadbot
\etape Le client remet le cadbot sur la grille.
\etape retour à l'étape \in[PC:RETOUR]
\stopcondition
\postcondition L'anomalie concernant ce cadbot sera transmise à la maintenance.
\stopalternatif
\startalternatif[PC:SE1]
\startcondition[Le cadbot ne detecte pas sa balise]
\etape Le client indique que le cadbot est en panne par validation du cas d'urgence "En panne".
\etape Le client remet la balise sur le cadbot
\etape Le client remet le cadbot sur la grille.
\etape retour à l'étape \in[PC:RETOUR]
\stopcondition
\postcondition par validation du cas d'urgence "En panne", L'annomalie de ce cadbot est transmis à la maintenance.
\stopalternatif
% les scénarii d'exceptions
\exception
Il n'y a plus de cadbot dans la parking
\stopCU
