% exemple de rédaction des scénarii associés à un CU
\subsection{Cas d'utilisation: «exemple» (A VIRER!)}

\startCU
\nom Consulter planning UE
\but Ce cas montre la suite de pas permettant à un enseignant de consulter le planning d'une UE
\acteur enseignant
\precondition l'enseignant est en charge d'au moins une UE
\declenchement l'enseignant a choisi de consulter le planning.
\auteur Alexandre G
\date 21/10/14

\nominal % scénario nominal
\startnominal
\etape[SA1] le système construit la liste des UE
\etape[RETOUR] l'enseignant choisit une UE
\etape le système construit le planning de l'UE
\etape[SA2] le système affiche le planning
\etape {\it fin du cas d'utilisation}
\stopnominal
\postcondition L'enseignant a consulté le planning de l'UE.

% les scénarii alternatifs
\alternatifs % ou \alternatif s'il n'y en a qu'un seul
\startalternatif[SA1] % ensemble des scénarii débutant à SA1
\startcondition[windaude crashes] 
  \etape le système est sous windaube et ne démarre pas
  \etape l'utilisateur s'énerve
  \etape il installe linux
  \etape retour à l'étape \in[RETOUR]
\stopcondition
\postcondition Tout est retourné à la normale
\stopalternatif

\startalternatif[SA2]
\startcondition[nombre de saisies incorrectes trop grand]
  \etape le système prévient le banquier
  \etape le système met fin à la session
\stopcondition
\startcondition[autre problème]
  \etape faire reset général
\stopcondition
\stopalternatif

% les scénarii d'exceptions
\exception % ou \exceptions s'il y en a plusieurs
Pas de scénario d'exception.
\stopCU
