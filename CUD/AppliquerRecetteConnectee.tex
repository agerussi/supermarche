\subsection{«AppliquerRecetteConnectée»}

\startCU
\nom AppliquerRecetteConnectée
\but Transférer les articles associés à une recette connectée dans la liste d'articles courante
\acteur le client
\precondition le client consulte une recette connectée 
\declenchement le client sélectionne l'option «appliquer la recette»
\auteur Alexandre G
\date 30/10/2014

\nominal % scénario nominal
\startnominal
\etape[ARC:DEBUT] le système affiche la page de mise à l'échelle de la recette
\etape[ARC:ZOOM] le client sélectionne «nombre de personnes» 
\etape le client saisit le nombre de personnes
\etape le client valide
\etape[ARC:VALIDATION] le système valide le choix du client
\etape[ARC:AFFICHAGE] le système affiche la liste des produits à acheter et les quantités associées
\etape[ARC:MODIFS] le client valide la liste
\etape le système affiche la page de gestion de la liste courante
\etape {\it fin du cas d'utilisation}
\stopnominal
\postcondition les articles ont été rajoutés à la liste courante du client

% les scénarii alternatifs
\alternatifs 
\startalternatif[ARC:ZOOM] 
\startcondition[le client sélectionne «appliquer un facteur»]
  \etape le client saisit le facteur d'échelle à appliquer (sans unité ou en pourcentages)
  \etape le client valide
  \etape retour à l'étape \in[ARC:VALIDATION]
\stopcondition
\stopalternatif

\startalternatif[ARC:VALIDATION] 
\startcondition[le facteur d'échelle est aberrant] 
  \etape le système affiche un avertissement expliquant pourquoi le choix semble aberrant
  \etape le client annule son choix (resp. confirme son choix)
  \etape retour à l'étape \in[ARC:DEBUT] (resp. \in[ARC:AFFICHAGE])
\stopcondition
\stopalternatif

\startalternatif[ARC:MODIFS] 
\startcondition[le client annule la liste] 
  \etape retour à l'étape \in[ARC:DEBUT]
\stopcondition
\startcondition[le client désélectionne un article] 
  \etape le système grise l'affichage de l'article et active l'option de sélection
  \etape retour à l'étape \in[ARC:MODIFS]
\stopcondition
\postcondition l'article désélectionné n'est plus dans la liste effective
\startcondition[le client sélectionne un article désélectionné] 
  \etape le système dé-grise l'affichage de l'article et active l'option de désélection
  \etape retour à l'étape \in[ARC:MODIFS]
\stopcondition
\postcondition l'article sélectionné est rajouté dans la liste effective
\startcondition[le client modifie un article] 
  \etape le système propose la liste des articles équivalents
  \etape le client en sélectionne un
  \etape le système retourne à la liste d'articles
  \etape retour à l'étape \in[ARC:MODIFS]
\stopcondition
\postcondition dans la liste, l'article a été remplacé par celui sélectionné par le client, et la quantité a été adaptée en conséquence
\stopalternatif

% les scénarii d'exceptions
\exception 
\startalternatif[ARC:ZOOM] 
\startcondition[le client sélectionne «annulation»]
  \etape le système retourne à consultation de la recette connectée
  \etape {\it fin du \cu} 
\stopcondition
\postcondition la liste d'articles courante est inchangée
\stopalternatif
\stopCU
