% exemple de rédaction des scénarii associés à un CU
\subsection{Cas d'utilisation: «AjouterArticleCadBot» }

\startCU
\nom AjouterArticleCadBot
\but Ce cas montre l'ajout d'un article dans le cadBot.
\acteur Client
\precondition Le client a pris son cadBot au cadPark.
\declenchement Le client est pret a deposer un produit dans le cadBot.
\auteur François Dubiez
\date 29/10/14

\nominal % scénario nominal
\startnominal
\etape[SA1] Le cadbot detecte le produit grace à sa puce RFID lorsque le client le passe devant le detecteur. 
\etape[RETOUR] Le cadbot émet un beep afin d'informer le client de la lecture du code RFID.
\etape Le cadbot compare le poids supposé de l'article avec le poids ajouté dans le caddie.
\etape L'objet est ajouté à la liste des objets achetés.
\etape le cadBot communique avec la marketTab afin d'indiquer le dernier ajout.
\etape {\it fin du cas d'utilisation}
\stopnominal
\postcondition L'objet a été ajouté a la liste des produits deposés dans le cadBot. Cela constitue la liste qui sera transmise lors du passage en caisse.

% les scénarii alternatifs
\alternatifs % ou \alternatif s'il n'y en a qu'un seul
\startalternatif[SA1] % ensemble des scénarii débutant à SA1
\startcondition[Lecture du code Barre] 
  \etape Certains produits vendu au poids ou a l'unité.(viande, poisson, fruits, legumes, etc...), necessite un pesage.
  \etape Lors du pesage du produit, la balance imprime avec un autocollant avec un code barre qui sera collé sur le sac contenant le produit.
  \etape Le client passe le code barre devant le lecteur optique du cadBot.
  \etape retour à l'étape \in[RETOUR]
\stopcondition
\postcondition L'objet est ajouté a la liste des achats potentiels.
\stopalternatif


% les scénarii d'exceptions
\exception % ou \exceptions s'il y en a plusieurs
\startalternatif[SA1]
\startcondition[ Ajout d'un produit non détécté ]
   \etape Le cadbot n'arrive pas à lire la puce RFID ou le code Barre.
   \etape Un son plus grave est émis pour alerter le clientainsi qu'un affichage d'un message d'alerte sur la markettab.
   \etape Le cadbot informera le portique de paiement qu'une anomalie a été détéctée.
\stopcondition
\postcondition L'anomalie sera transmise au portique de paiement.
\stopalternatif

\stopCU
