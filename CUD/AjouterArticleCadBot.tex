% exemple de rédaction des scénarii associés à un CU
\subsection{Cas d'utilisation: «AjouterArticleCadBot» }

\startCU
\nom AjouterArticleCadBot
\but Ce cas montre l'ajout d'un article dans le cadBot.
\acteur Client
\precondition Le client a pris son cadBot au cadPark.
\declenchement Le client est pret à deposer un produit dans le cadBot.
\auteur François Dubiez
\date 29/10/14

\nominal % scénario nominal
\startnominal
\etape[AAC:SA1] Le cadbot détecte le produit grace à sa puce RFID lorsque le client le passe devant le détecteur. 
\etape[AAC:RETOUR] Le cadbot émet un beep afin d'informer le client de la lecture du code RFID.
\etape[AAC:SE1] Le cadbot valide l'ajout de l'objet en comparant le poids obtenu via le tag RFID et  le poids ajouté dans le caddie.
\etape {\it fin du cas d'utilisation}
\stopnominal
\postcondition Le CadBot ajoute l'objet à la liste des produits deposés.

% les scénarii alternatifs
\alternatifs % ou \alternatif s'il n'y en a qu'un seul
\startalternatif[AAC:SA1] % ensemble des scénarii débutant à SA1
\startcondition[Lecture du code Barre] 
  \etape[AAC:SE1] Le client passe le code barre devant le lecteur optique du cadBot.
  \etape retour à l'étape \in[AAC:RETOUR]
\stopcondition
\postcondition Le CadBot a correctement lu le code barre.
\stopalternatif

\startalternatif[AAC:SE1]
\startcondition[variation de Poids non detecté]
   \etape Le cadbot ne détecte pas la variation de poids dans le caddie.
   \etape Le cadBot émet un son plus grave émis pour alerter le client.
\stopcondition
\postcondition L'anomalie concernant cet objet sera transmise au portique de paiement.
\stopalternatif

\startalternatif[AAC:SA1]
\startcondition[Ajout d'un produit non détécté]
   \etape Le cadbot n'arrive pas à lire la puce RFID ou le code Barre.
   \etape Le cadBot émet un son plus grave émis pour alerter le client.
   \etape Le cadbot affecte le poids a un objet inconnu afin de continuer à suivre l'evolution du poids en fonction des objets.
\stopcondition
\postcondition L'anomalie sera transmise au portique de paiement.
\stopalternatif

% les scénarii d'exceptions
\exception % ou \exceptions s'il y en a plusieurs
Aucun scénario d'exception
\stopCU
