\subsection{«RécuperationCadBotEnPanne»}
\TODO\ A remanier et discuter en profondeur: 
* trop d'hypothèses ou d'options peu réalistes
* description trop macroscopique des interactions
* CU non prioritaire car lié au personnel et non au client...
\startCU
\nom RécuperationCadBotEnPanne
\but Ce cas montre la suite de pas permettant au personnel de récupérer et remettre le cadBot en panne au cadPark.
\acteur Personnel du magasin.
\precondition Le personnel a pris un marketTab.
\auteur Raissa Mbabazi Simbi, François Dubiez
\date 7/11/14, 9/11/14
\nominal % scénario nominal
\startnominal
\etape[RCP:RETOUR1] la marketTab indique le chemin pour aller à la dernière position connue du cadBot en panne.
\etape[RCP:SE1] Le personnel trouve le cadBot, scanne son code-barre et le ramène au cadPark.
\etape Le système reçoit l'information de la marketTab et marque le cadBot comme inopérant.
\etape Le personnel scanne l'emplacement du cadPark pour le communiquer au système.
\etape [RCP:RETOUR] Le CadBot est sur un emplacement et commence à se recharger.
\etape [RCP:ANO] Le cadBot lance une vérifcation et envoie le résultat au système. 
\etape Le système informe le personnel de maintenance du l'état du cadBot.
\etape[RCP:FIN] {\it fin du cas d'utilisation}
\stopnominal
\postcondition Le CadBot est indiqué localisé et inopérant dans le système.
.
% les scénarii alternatifs
\alternatifs
\startalternatif[RCP:SE1]
\startcondition[Le code barre du cadBot n'est pas lisible]
\etape Le personnel entre le numero du cadBot dans le formulaire cadBot en panne de la marketTab.
\etape Le personnel ramène le cadBot au cadPark.
\etape Retour à l'étape \in[RCP:RETOUR]
\stopcondition
\postcondition Le système a identifié le cadBot en panne et a mis à jour son état.
\stopalternatif


\startalternatif[RCP:ANO]
\startcondition [ Aucune communication n'a lieu entre le cadBot et le système ]
\etape Le système informe la maintenance qu'aucune communication n'est possible avec ce cadBot.
\etape retour à l'étape \in[RCP:FIN]
\stopcondition
\postcondition Le sytème marque ce cadBot comme nécéssitant de la maintenance et connait son emplacement.
\stopalternatif


% les scénarii d'exceptions
\exception
Le cadBot n'est pas à l'emplacement indiqué, ni dans les environs.
\stopCU
