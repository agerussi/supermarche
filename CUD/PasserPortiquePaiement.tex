% exemple de rédaction des scénarii associés à un CU
\subsection{«PasserPortiquePaiement»}

\startCU
\nom PasserPortiquePaiement
\but le CadBot communique à la caisse la liste des objets qu'il contient
\acteur client, personnel du magasin
\precondition le portique de paiement est en service et libre (indiqué par une lumière verte), ses barrières de sortie sont bloquées
\declenchement le système détecte la présence du cadbot entre ses panneaux
\auteur François Dubiez, Alexandre Gerussi
\date 30/10/14, révision 08/10/14

\nominal % scénario nominal
\startnominal
\etape le système allume la lumière orange du portique
\etape[PPP:PASCOM] le système établit une communication avec le cadBot
\etape[PPP:ANOMALIES] le CadBot envoie au système la liste détaillée des articles qu'il contient
\etape [PPP:PAIEMENT] le système établit la facture et initialise la caisse du portique pour le paiement
\etape le scénario «PayerCourses» est enclenché au niveau de la caisse du portique
\etape le système détecte la fin normale du scénario «PayerCourses»
\etape le système débloque les barrières du portique
\etape le client sort du portique par les barrières
\etape[PPP:DETECTION] le système détecte le passage du client dans les barrières
\etape[PPP:RESET] le système bloque les barrières
\etape le système allume la lumière verte du portique
\etape {\it fin du cas d'utilisation}
\stopnominal
\postcondition le portique est prêt pour l'accueil d'un nouveau cadBot

% les scénarii alternatifs
\alternatifs % ou \alternatif s'il n'y en a qu'un seul

\startalternatif[PPP:ANOMALIES]
\startcondition[des anomalies avaient été enregistrées durant les achats]
  \etape Le CadBot envoie au système la liste des anomalies détectées.
  \etape Le système allume la lumière rouge du portique
  \etape Un personnel du magasin essaye de traiter les anomalies avec le client. En cas d'échec le client est redirigé vers une caisse à hôtesse comme dans le scénario d'exception. {\it Rq: cette étape mériterait un approfondissement, par exemple en fonction du genre de pb rencontré durant les achats\dots}
  \etape retour à l'étape \in[PPP:PAIEMENT]
\stopcondition
\postcondition toutes les anomalies sont réglées, le système a identifié l'ensemble des articles
\stopalternatif

\startalternatif[PPP:DETECTION]
\startcondition[le système détecte que le cadBot a été retiré des panneaux mais pas un passage dans les barrières dans un délai raisonnable (30 secondes ?)]
  \etape Le système allume la lumière rouge en mode clignotant 
  \etape Un personnel du magasin essaye de traiter le problème en invitant le client à sortir par un passage «sans achats»
  \etape Le personnel scanne sa carte au niveau de la caisse du portique
  \etape le système affiche le menu de maintenance
  \etape le personnel sélectionne «passage forcé»
  \etape le système affiche l'écran normal de la caisse
  \etape retour à l'étape \in[PPP:RESET]
\stopcondition
\stopalternatif

% scénarii d'exception
\exception
\startalternatif[PPP:PASCOM] 
\startcondition[Pas de communication cadBot / Portique]
  \etape le système allume une lumière rouge au niveau du portique
  \etape le système invite le client à se diriger vers une caisse à hôtesse
  \etape le client retire le cadBot des panneaux du portique
  \etape le système détecte le départ du cadBot
  \etape le système allume la lumière verte du portique
  \etape {\it fin du \cu}
\stopcondition
\postcondition le portique est opérationnel pour un nouveau client
\stopalternatif

\stopCU
