% exemple de rédaction des scénarii associés à un CU
\subsection{Cas d'utilisation: «PasserPortiquePAiement» }

\startCU
\nom PasserPortiquePaiement
\but le CadBot se positionne sur l'emplacement dédié en caisse et transfert la liste des objets présents dans le cadBot
\acteur client
\precondition Le client a fait ses courses avec un cadBot.
\declenchement le client a terminé ses courses et se dirige vers le portique de paiement.
\auteur François Dubiez
\date 30/10/14

\nominal % scénario nominal
\startnominal
\etape[SA1] Le client se dirige vers la caisse comportant un portique de paiement.
\etape[RETOUR] le CadBot est dans l'emplacement dédié. Le portique a détecté le cadBot et une communication s'engage entre le portique et le cadBot.
\etape[SA2] Le CadBot transmet le liste des objets détectés.
\etape [RETOUR2]{\it fin du cas d'utilisation}
\stopnominal
\postcondition Le client est prêt a procéder au paiement.

% les scénarii alternatifs
\alternatifs % ou \alternatif s'il n'y en a qu'un seul
\startalternatif[SA1] % ensemble des scénarii débutant à SA1
\startcondition[ModeManuel] 
  \etape le cadBot est placé par le client sur l'emplacement adéquat.
  \etape  Un marquage au sol est destiné a aider le client a positionner correctement le cadBot dans le portique.
  \etape retour à l'étape \in[RETOUR]
\stopcondition
\postcondition Le cadBot est en mesure de transmettre la liste des produits détectés.
\stopalternatif

\startalternatif[SA2]
\startcondition[Anomalie de poids]
  \etape Le CadBot détecte une différence de poids trop grande entre la somme des poids des articles détectés et le poids total détecté.
  \etape Le cadBot informe le portique de cette anomalie.
  \etape Un personnel du magasin est informé de l'anomalie et vient aider le client a identifier le problème.
  \etape Le problème étant identifié, le correctif est appliqué (suppression de l'article non présent, ajout de l'article non détecté a la liste....) par le personnel du magasin.
  \etape retour à l'étape \in[RETOUR2]
\stopcondition
\stopalternatif

% les scénarii d'exceptions
\exception % ou \exceptions s'il y en a plusieurs
\startalternatif[RETOUR] 
\startcondition[Pas de communication]
  \etape Le cadBot n'est pas en mesure de communiquer avec le portique de paiement.
  \etape Le cadBot émet en beep afin d'attirer l'attention du client sur la marketTab.
  \etape Le cadBot envoie un message d'alerte sur la marketTab informant le client.
  \etape Le cadBot envoie par wifi une alerte au système informatique du magasin afin d'alerter le personnel.
 \etape La détection des produits se fera en mode dégradé, a savoir en sortant et bipant tout les produits.
\stopcondition
\postcondition Le système central est informé de l'impossibilité de poursuivre la procédure.
\stopalternatif
\stopCU
