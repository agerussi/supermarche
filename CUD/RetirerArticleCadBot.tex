% exemple de rédaction des scénarii associés à un CU
\subsection{«RetirerArticleCadBot»}

\startCU
\nom RetirerArticleCadBot
\but Ce cas montre le retrait d'un article du cadBot.
\acteur Client
\precondition Le client a pris son cadBot au cadPark.
\declenchement Le client retire un produit du cadBot.
\auteur François Dubiez
\date 30/10/14

\nominal % scénario nominal
\startnominal
\etape Le cadBot détecte une variation de poids.
\etape le cadBot attend l'identification d'un article
\etape[RAC:IDENTIFICATION] le client passe l'article devant le détecteur RFID.
\etape[RAC:SA1] Le cadBot émet un beep afin d'informer de l'identification du produit.
\etape[RAC:SA3] Le cadBot valide la suppression de l'article en comparant la variation de poids du caddie.
\etape {\it fin du cas d'utilisation}
\stopnominal
\postcondition Le produit est retiré de la liste des objets déposés.

% les scénarii alternatifs
\alternatif % ou \alternatif s'il n'y en a qu'un seul
\startalternatif[RAC:IDENTIFICATION] % ensemble des scénarii débutant à SA1
\startcondition[Lecture du code Barre] 
  \etape Le client passe le code barre devant le lecteur optique du cadBot.
  \etape retour à l'étape \in[RAC:SA1]
\stopcondition
\postcondition La cadBot a réussi à lire le code Barre.
\stopalternatif

\exceptions
\startalternatif[RAC:IDENTIFICATION]
\startcondition[le client n'identifie pas l'article]
   \etape Le cadBot émet un son plus grave pour alerter le client.
   \etape Le cadbot garde en mémoire la variation de poids pour la transmettre au portique ainsi que le poids total actuel afin de poursuivre le suivi.
\stopcondition
\postcondition Le cadBot transmettra l'anomalie au portique de paiement.
\stopalternatif

\startalternatif[RAC:SA1]
\startcondition[Erreur de lecture Tag RFID/code barre]
   \etape Le cadBot émet un son plus grave pour alerter le client.
   \etape Le cadbot garde en mémoire la variation de poids pour la transmettre au portique ainsi que le poids total actuel afin de poursuivre le suivi.
\stopcondition
\postcondition Le cadBot transmettra l'anomalie au portique de paiement.
\stopalternatif

\startalternatif[RAC:SA3]
\startcondition[variation de Poids non detecté]
   \etape Le cadBot émet un son plus grave pour alerter le client.
\stopcondition
\postcondition Le cadBot transmettra une alerte au portique concernant cet objet en indiquant que la diminution de poids n'a pas été détéctée.
\stopalternatif

\stopCU
