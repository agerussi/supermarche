% exemple de rédaction des scénarii associés à un CU
\subsection{Cas d'utilisation: «RetirerArticleCadBot» }

\startCU
\nom RetirerArticleCadBot
\but Ce cas montre le retrait d'un article du cadBot.
\acteur Client
\precondition Le client a pris son cadBot au cadPark.
\declenchement Le client est pret a retirer un produit du cadBot.
\auteur François Dubiez
\date 30/10/14

\nominal % scénario nominal
\startnominal
\etape[SA1] Le cadbot detecte le produit grace a sa puce RFID lorsque le client le passe devant le detecteur. 
\etape[RETOUR] Le cadbot emet un beep afin d'informer le client de la lecture du code RFID.
\etape Le cadbot a detecté une diminution du poids correspondant a l'article identifié précédemment.
\etape L'objet est ôté de la liste des objets achetés.
\etape {\it fin du cas d'utilisation}
\stopnominal
\postcondition L'objet a été ôté de la liste des produits deposés dans le cadBot. Cela constitue la liste qui sera transmise lors du passage en caisse.

% les scénarii alternatifs
\alternatifs % ou \alternatif s'il n'y en a qu'un seul
\startalternatif[SA1] % ensemble des scénarii débutant à SA1
\startcondition[Lecture du code Barre] 
  \etape Certains produits vendu au poids ou a l'unité.(viande, poisson, fruits, legumes, etc...), necessite un pesage.
  \etape Lors du pesage du produit, la balance imprime avec un autocollant avec un code barre qui sera collé sur le sac contenant le produit.
  \etape Le client passe le code barre devant le lecteur optique du cadBot.
  \etape retour à l'étape \in[RETOUR]
\stopcondition
\postcondition L'objet est ôté de la liste des achats potentiels.
\stopalternatif


% les scénarii d'exceptions
\exception % ou \exceptions s'il y en a plusieurs
La puce RFID n'est pas reconnu/ Le code Barre n'a pu etre lu.
Lors du passage en caisse, il conviendra d'appeler une hotesse de caisse afin de traiter l'anomalie manuellement.
\stopCU
