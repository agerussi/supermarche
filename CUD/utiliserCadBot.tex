% exemple de rédaction des scénarii associés à un CU
\subsection{Cas d'utilisation: «utiliserCadBot» }

\startCU
\nom utiliserCadBot
\but Ce cas montre la suite de pas permettant à un client de faire ses courses avec le cadbot
\acteur client
\precondition le client a pris son cadBot au cadPark
\declenchement Le client a choisi de faire ses achats dans le magasin.
\auteur François Dubiez
\date 29/10/14

\nominal % scénario nominal
\startnominal
\etape[SA1] Le CadBot suit le client dans le magasin. 
\etape[RETOUR] Le client se déplace dans les allées à la recherche des produits de sa liste de course.
\etape Lorsque le client ajoute deux objets dans le cadBot, le CU AjouterArticleCadbot (extend) est utilisé.
\etape Le client se rapelle qu'il n'a besoin que d'un objet, il le retire donc du cadBot,(extend) le CU RetirerArticleCadBot est lancé.
\etape Lorsque le client a obtenu les produits qu'il cherchait, il se dirige vers le portique de paiement. (include PasserPortiquePaiement)
\etape {\it fin du cas d'utilisation}
\stopnominal
\postcondition Le client a terminé ses courses et le cadBot se trouve dans le portique de paiement.

% les scénarii alternatifs
\alternatifs % ou \alternatif s'il n'y en a qu'un seul
\startalternatif[SA1] % ensemble des scénarii débutant à SA1
\startcondition[ModeAutonome] 
  \etape le client n'a pas pris la balise du cadbot.
  \etape le client pousse le cadbot au lieu que ce dernier le suive.
  \etape retour à l'étape \in[RETOUR]
\stopcondition
\postcondition Le client pousse le cadBot.
\stopalternatif


% les scénarii d'exceptions
\exception % ou \exceptions s'il y en a plusieurs
Le client n'a pas fait d'achat, se rend au portique de paiement avec un cadBotVide.
le portique de paiement detecte un chariot vide et permet au client d'aller déposer le cadBot au cadPark.
\stopCU
