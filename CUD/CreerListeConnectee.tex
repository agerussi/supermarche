\subsection{Cas d'utilisation: CréerListeConnectée}

\startCU
\nom CréerListeConnectée
\but Ce cas d’utilisation correspond à la création d’une nouvelle liste connectée, vierge au départ, qui pourra ensuite être complétée.
\acteur client
\precondition Le client est connecté sur le site ou en possession d'une markettab
\declenchement le client clique sur nouvelle liste de course dans le menu.
\auteur François D.
\date 03/11/14

\nominal % scénario nominal
\startnominal
\etape[CLC:ANNUL] Le système demande le nom de la liste et propose également d'annuler la création de liste.
\etape[CLC:NOMINVAL] Le système valide le nom de la liste fournie par le Client.
\etape[CLC:RETOURDL] {\it fin du cas d'utilisation}
\stopnominal
\postcondition Le système a enregistré une nouvelle liste de courses.

% les scénarii alternatifs
\alternatifs % ou \alternatif s'il n'y en a qu'un seul
\startalternatif[CLC:NOMINVAL] % ensemble des scénarii débutant à SA1
\startcondition[Nom invalide]
  \etape Le système informe le client que le nom est incorrect.
  \etape Le système demande à nouveau le nom de la liste et propose également d'annuler la création de liste.
  \etape Le client entre un nom correct (resp. annule)
  \etape retour à l'étape \in[CLC:NOMINVAL](resp. fin du cas d'utilisation.\in[CLC:RETOURDL])
\stopcondition
\stopalternatif

% les scénarii d'exceptions
\exception % ou \exceptions s'il y en a plusieurs
\startalternatif[CLC:ANNUL]
\startcondition[Annulation de la création]
  \etape Le client clique sur le bouton annuler.
  \etape Le système met fin à la creation de liste et fin du cas d'utilisation.\in[CLC:RETOURDL].
\stopcondition
\postcondition La liste de courses est inchangée
\stopalternatif
\stopCU
