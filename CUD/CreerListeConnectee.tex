\subsection{Cas d'utilisation: CréerListeConnectée}

\startCU
\nom CréerListeConnectée
\but Ce cas d’utilisation correspond à la création d’une nouvelle liste connectée, vierge au départ, qui pourra ensuite être complétée.
\acteur client
\precondition Le client est connecté sur le site ou en possession d'une markettab
\declenchement le client désire créer une nouvelle liste de course.
\auteur François D.
\date 03/11/14

\nominal % scénario nominal
\startnominal
\etape Le client clique Liste de courses du menu.
\etape Le système affiche dans la barre du menu 'Nouvelle liste' puis la liste des courses deja enregistrées dans la boite située en dessous à gauche.
\etape Le client clique sur Nouvelle liste.
\etape[ANNUL] Le système demande le nom de la liste et propose également d'annuler la création de liste.
\etape[NOMINVAL] Le système valide le nom de la liste fournie par le Client.
\etape[RETOUR] {\it fin du cas d'utilisation}
\stopnominal
\postcondition Le système a enregistré une nouvelle liste de courses.

% les scénarii alternatifs
\alternatifs % ou \alternatif s'il n'y en a qu'un seul
\startalternatif[NOMINVAL] % ensemble des scénarii débutant à SA1
\startcondition[Nom invalide]
  \etape le système demande à nouveau le nom de la liste et propose également d'annuler la création de liste.
  \etape Le client entre un nom correct.
  \etape Le système valide le nom de la liste.
  \etape retour à l'étape \in[RETOUR]
\stopcondition
\stopalternatif

% les scénarii d'exceptions
\exception % ou \exceptions s'il y en a plusieurs
\startalternatif[ANNUL]
\startcondition[Annulation de la création]
  \etape Le client clique sur le bouton annuler.
  \etape le système met fin à la creation de liste et revient à l'affichage de nouvelle liste et de la liste de courses.
\stopcondition
\postcondition La liste de courses est inchangée
\stopalternatif
\stopCU
