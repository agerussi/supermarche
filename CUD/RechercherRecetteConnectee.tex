\subsection{«RechercherRecetteConnectée»}

\startCU
\nom RechercherRecetteConnectée
\but se rendre sur la page de consultation d'une recette connectée
\acteur le client
\precondition le client est connecté sur le site ou en possession d'une markettab
\declenchement le client désire consulter une recette connectée
\auteur Alexandre G
\date 30/10/2014

\nominal % scénario nominal
\startnominal
\etape[RRC:DEBUT] le client choisit un article (par clic à partir d'une liste ou scan en magasin)
\etape le système affiche la page de consultation de l'article
\etape le client choisit l'option «recettes associées»
\etape le système affiche la liste des recettes associées à l'article 
\etape[RRC:SELECT] le client en sélectionne une
\etape le système affiche la page consultation de la recette
\etape {\it fin du cas d'utilisation}
\stopnominal
\postcondition le client a consulté une recette

% les scénarii alternatifs
\alternatifs 
\startalternatif[RRC:DEBUT] 
\startcondition[le client sélectionne l'option «rechercher recettes»] 
  \etape le système affiche la liste des recettes connues par ordre alphabétique
  \etape retour à l'étape \in[RRC:SELECT]
\stopcondition
\stopalternatif

\startalternatif[RRC:SELECT] 
\startcondition[le client saisit quelques lettres dans le cadre «filtre»]
  \etape le système filtre les recettes affichées à l'aide du motif tapé par le client
  \etape retour à l'étape \in[RRC:SELECT]
\stopcondition
\stopalternatif
\stopCU
