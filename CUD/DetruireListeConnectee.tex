\subsection{Cas d'utilisation: DetruireListeConnectee}

\startCU
\nom DetruireListeConnectee
\but Détruire une/des liste(s) précédemment crée(s).
\acteur Le client
\precondition Le client consulte une liste connectée.
\declenchement le client sélectionne l'option «supprimer la liste»
\auteur François D.
\date 05/11/2014

\nominal % scénario nominal
\startnominal
\etape[DEJALISTE] Le client sélectionne Supprimer une/des listes dans le menu.
\etape Le système propose au client une série de liste de courses sélectionnables pour la suppression avec l'action OK ou ANNULER.
\etape[ANNULATION] Le client sélectionne une ou plusieurs listes à supprimer puis clique sur OK.
\etape[VALIDATION] Le système valide le choix du client
\etape[AFFICHAGE] Le système supprime la/les liste(s) sélectionnée(s).
\etape[FIN] {\it fin du cas d'utilisation}
\stopnominal
\postcondition la/les listes ont ont été supprimées de la liste courante de liste de courses

% les scénarii alternatifs
\alternatifs 
\startalternatif[DEJALISTE] 
\startcondition[le client se trouve déja dans une liste de courses et désire la supprimer.Le système reconnait qu'une liste est en cours d'édition et le menu 'Supprimer une/des listes' est devenu 'Supprimer cette liste'.]
  \etape Le client clique sur 'Supprimer cette liste'
  \etape Le système demande confirmation.
  \etape Le client confirme la suppression.
  \etape Retour à l'étape \in[VALIDATION]
\stopcondition
\stopalternatif

% les scénarii d'exceptions
\exception 
\startalternatif[ANNULATION] 
\startcondition[le client sélectionne «ANNULATION»]
  \etape Le système retourne à l'affichage de la série de liste de courses.
  \etape {\it fin du \cu} 
\stopcondition
\postcondition La liste de courses est inchangée
\stopalternatif
\stopCU
