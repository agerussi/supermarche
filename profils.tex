Nos scénarios concrets mettent en scène diverses personnes dont nous proposons ici un rapide profil.
\blank[big]

\startfiguretext
[left, none]{}{\externalfigure[PHOTOS-PROFIL/PROF1_tsong.png][width=1.5cm]}
Congcong Xu, étudiant.\crlf
Congcong étudie à l'université de Lille 1, c'est la prémière année qu'il étudie en France, il trouve qu'il doit passer beaucoup de temps en sa étude, parce qu'il a mal en Français et M1 d'informatique  est aussi un peu difficile pour lui aussi. Donc il préfère bien faire des courses sur Internet pour gagner du temps, maintenant, il a habitude de choisir les choses sur le site de son futurmarket pour la semaine, et il pense que ce moyen d'achat est trés facile pour lui, il aime bien, il veut aussi le recommander à ses amis.
\stopfiguretext

\startfiguretext
[left, none]{}{\externalfigure[PHOTOS-PROFIL/PROF2_Alex.png][width=1.5truecm]}
Alexandre Gerussi, professeur de Mathématiques.\crlf
Alexandre est un prof trés diligent et dispos, il enseigne la mathématique et aussi apprendre M1 d'informatique, il y a un chinois dans son groupe, il doit passer beaucoup de temps pour l'expliquer et l'instruire, c'est très fatigant pour lui, il dépense beaucoup d'énergie tous les jours, et il n'a pas beaucoup de temps pour manger des alimentations différentes. Il sait bien c'est pas bon pour la santé, donc il ira acheter des alimentations sur Internet par quelques mois, il aime surtout les recettes sur les alimentations, il pense que c'est trés humain, et il dit que les recettes sont pas mal, il bénéficie beaucoup.
\stopfiguretext

\startfiguretext
[left, none]{}{\externalfigure[PHOTOS-PROFIL/PROF3_Raissa.png][width=1.5truecm]}
Raïssa Mbabazi Simbi, étudiante et travaillant en parallèle.\crlf
Raissa doit travailler jusqu'à trés tard tous les jours, et elle est une jeune mère, elle a deux petits enfants, son homme doit passer aussi beaucoup de temps en travail et son homme ne sait pas comment faire la cuisine, il a essayé de faire, mais les enfants n'aiment pas le gout des alimentations qu'il fait. Donc Raissa doit faire des courses et faire la cuisine aprés le travail, grace à le “supermarche”, elle peut acheter des alimentations sur Internet rapidement, et aussi les recettes l'aident beaucoup, parce qu'elle ne sait pas bien comment faire la cuisine. Elle est trés contentes, elle dit que ses enfants aiment bien sa cusine.
\stopfiguretext

\startfiguretext
[left, none]{}{\externalfigure[PHOTOS-PROFIL/PROF4_mickaella.png][width=1.5truecm]}
Mickaella Dubiez, employée.\crlf
Mickaella est employé dans le magasin, elle s'ocuppe de servir à des clients qui ont des problèmes sur le magasin, par exemple, un client ne peut pas trouver le produit, elle va l'aider. Et aussi elle s'ocuppe de mettre en rayon des produits. Elle trouve que c'est un facile travail malgré elle n'a pas beaucoup de salaire, et elle va continuer de faire son travail malgré pour l'autre personnes, c'est un travail un peu ennuyé. Elle dit que elle aime son travail et amie servir à des personnes qui demandent les aides.
\stopfiguretext

\startfiguretext
[left, none]{}{\externalfigure[PHOTOS-PROFIL/PROF5_gaetan.png][width=1.5truecm]}
Gaetan Laajine, chef de rayon au \fm.\crlf
Gaetan est employé dans le magasin, comme Mickaella, il est petit amis de Mickaella, il s'ocuppe de reapprovisionnement du magasin, grace à l'application WoodStock, il peut savoir bien quels produits doit etre fourni, et l'application WoodStock peut aussi l'aider à acheter des produits automatiquement. C'est un facile boulot pour lui, mais c'est un important boulot pour le magasin, il dit.
\stopfiguretext

\startfiguretext
[left, none]{}{\externalfigure[PHOTOS-PROFIL/PROF6_mariline.png][width=1.5truecm]}
Mariline Gerussi, employée de \fm.\crlf
Mariline est employé dans le magasin aussi, elle commence son poste du matin à 09h00. Elle s'occupe de scanner des achats des clients. Maintenant il n'y a pas beaucoup de personnes qui paie par elle, ça lui permet de prendre du repos après un bout de temps, elle jouit de son travail, elle pense que prenant du repos est meillieur que prenant du café. Après le boulot, elle dépose le tiroir caisse dans le local sécurisé, se change et repart chez elle en badgeant avant de partir. Elle merci bien de developpement de technologie.
\stopfiguretext
