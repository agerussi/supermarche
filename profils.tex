Nos scénarios concrets mettent en scène diverses personnes dont nous proposons ici un rapide profil.
\blank[big]

\startfiguretext
[left, none]{}{\externalfigure[PHOTOS-PROFIL/PROF1_tsong.png][width=1.5cm]}
Congcong Xu, étudiant.\crlf
Aime faire ses courses sur internet.....
\stopfiguretext

\startfiguretext
[left, none]{}{\externalfigure[PHOTOS-PROFIL/PROF2_Alex.png][width=1.5truecm]}
Alexandre Gerussi, professeur de Mathématiques.\crlf
Aimerait soulager sa femme au maximum de la corvée du repas, et aussi varier les plaisirs culinaires au maximum: il apprécie particulièrement les recettes connectées du \fm.
\stopfiguretext

\startfiguretext
[left, none]{}{\externalfigure[PHOTOS-PROFIL/PROF3_Raissa.png][width=1.5truecm]}
Raïssa Mbabazi Simbi, étudiante et travaillant en parallèle.\crlf
Elle doit travailler jusqu'à très tard, donc elle a besoin de faire ses courses sur Internet après le travail. 
Elle ne sait pas bien comment faire la cuisine.
Elle aime bien les recettes que le \fm\ offre.
\stopfiguretext

\startfiguretext
[left, none]{}{\externalfigure[PHOTOS-PROFIL/PROF4_mickaella.png][width=1.5truecm]}
Mickaella Dubiez, employée.\crlf
Elle préfère acheter des choses sur Internet quand elle demande.
\stopfiguretext

\startfiguretext
[left, none]{}{\externalfigure[PHOTOS-PROFIL/PROF5_gaetan.png][width=1.5truecm]}
Gaetan Laajine, chef de rayon au \fm.\crlf
Il a habitude de s'occuper du stock de son rayon, «papeterie».
\stopfiguretext

\startfiguretext
[left, none]{}{\externalfigure[PHOTOS-PROFIL/PROF6_mariline.png][width=1.5truecm]}
Mariline Gerussi, employée de \fm.\crlf
Elle est employée du supermarché, elle sert à des clients, quand les clients ont fini leurs courses, elle leur aide à scanner les achats.
\stopfiguretext
