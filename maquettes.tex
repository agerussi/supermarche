\chapter{Maquettes interactives}

Nous avons développé quelques maquettes interactives qui reprennent certains de nos scénarios et commencent à préciser les écrans du futur système.

%La \in{figure}[FIG:ECRANS] décrit les liens entre les différents écrans sous la forme d'un diagramme d'état UML.
%\placefigure[here][FIG:ECRANS]{La structure de l'UI}{\externalfigure[fichier.pdf][width=\hsize]}

Les liens suivants permettent d'explorer dynamiquement les scénarios sur un navigateur web:
\useURL[BE][HTML/borne_eureka.html][][«UtiliserBorneEurêka»]
\useURL[GUIDE][HTML/guidage_markettab.html][]["UtiliserGuidageMarketTab"]
\useURL[CHREC][HTML/chercher_recette.html][]["RechercherRecetteConnectée"]
\startitemize
\item le \cu\ \from[BE]; mode d'emploi:
    \startitemize[n]
    \item on écrit quelques mots-clés dans la zone prévue à cet effet (ici il est supposé que le client prononce les mots «petits suisses»), puis on appuie sur la petite loupe pour valider;
    \item le système propose alors une liste de produits associés; on en sélectionne un
    \item en cliquant sur «plan», le système localise le produit sur le plan, ainsi que la position actuelle du client (devant la borne).
    \stopitemize
\item le \cu\ \from[GUIDE] pour se faire guider à partir d'une liste de courses; mode d'emploi:
    \startitemize[n]
\item le premier écran est l'écran d'accueil par lequel le client s'identifie, soit par scan de sa carte, soit par saisie de son code client;
\item une fois reconnu, le client se retrouve dans la page de consultation de ses listes connectées personnelles; il en sélectionne une, dont le détail s'affiche;
\item il peut ensuite sélectionner un type de parcours (seul le rapide est implémenté);
\item un plan du magasin apparaît alors avec le parcours optimal tracé, ainsi que les emplacements exacts des produits cherchés.
    \stopitemize

\item le \cu\ \from[CHREC] pour rechercher une recette a partir de son nom ou d'un produit; mode d'emploi:
    \startitemize[n]
\item le premier écran est l'écran de recherche de recette via un nom de recette ou un produit;
\item La recherche validée, le détail du produit est affiché.
\item En cliquant sur Rechercher les Recettes, la liste des recettes est affichée.
\item Le client choisit alors parmi les recettes et se retrouve dans la page de la recette proprement dite.
\item Il peut ensuite procéder à la mise à l'échelle de la recette.
\item L'écran suivant indique alors la liste et la quantité d'articles nécessaires à la confection de la recette. 
Le client peut supprimer l'article ou modifier les quantités proposées.
\item Le client après validation de la liste reçoit la confirmation de l'ajout de sa liste des articles à sa liste de courses.
    \stopitemize
\stopitemize
