\chapter{Maquettes interactives}

Nous avons développé quelques maquettes interactives qui reprennent certains de nos scénarios et commencent à préciser les écrans du futur système.

%La \in{figure}[FIG:ECRANS] décrit les liens entre les différents écrans sous la forme d'un diagramme d'état UML.
%\placefigure[here][FIG:ECRANS]{La structure de l'UI}{\externalfigure[fichier.pdf][width=\hsize]}

Les liens suivants permettent d'explorer dynamiquement les scénarios sur un navigateur web:
\useURL[BE][HTML/borne_eureka.html][][«UtiliserBorneEurêka»]
\startitemize
\item le \cu\ \from[BE]; mode d'emploi:
    \startitemize[n]
    \item on écrit quelques mots-clés dans la zone prévue à cet effet (ici il est supposé que le client prononce les mots «petits suisses»), puis on appuie sur la petite loupe pour valider;
    \item le système propose alors une liste de produits associés; on en sélectionne un
    \item en cliquant sur «plan», le système localise le produit sur le plan, ainsi que la position actuelle du client (devant la borne).
    \stopitemize
\stopitemize
