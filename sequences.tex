\chapter{Diagrammes de séquences}

Dans cette section nous présentons les diagrammes de séquence (DS) reliés à quelques-uns des \cu\ les plus emblématiques de notre application.\crlf
Ces diagrammes ont été élaborés selon le pattern Modèle Vue Contrôleur. 
Étant donné qu'il existe plusieurs moutures différentes de ce pattern (qui diffèrent notamment par les relations entre le modèle et les vues), nous avons formulé notre propre synthèse. 
Nous la présentons pour référence en annexe de ce document (\in{figure}[FIG:MVC]).

\startitemize
\item la \in{figure}[DS:BE] correspond au \cu\ «UtiliserBorneEuréka», voir \in{section}[SS:CUD:UBE].
Le diagramme de classes détaillé correspondant est à la \in{figure}[DC:BE].
Les classes faisant partie de l'UI sont les «vues» et possèdent leur propre fil d'exécution afin de rester réactives aux sollicitations du client.
Par exemple, l'objet \type{carteUI} répond à des touchers de l'utilisateur pour faire défiler ou zoomer la carte (ces interactions ne sont pas représentées sur le diagramme).

\item les \in{figures}[DS:AR1] et \in[DS:AR2] correspondent au \cu\ «AppliquerRecetteConnectée», voir \in{section}[SS:CUD:AR].
Le diagramme de classes détaillé correspondant est à la \in{figure}[DC:AR].
Ce diagramme de séquence ne reprend pas en intégralité l'ensemble des scénarios alternatifs du \cu; en effet cela aurait compliqué outre mesure le diagramme déjà trop chargé comme cela, sans rien apporter de neuf au niveau conception.

\item la \in{figure}[DS:AAC] correspond au \cu\ «AjouterArticleCadbot», voir \in{section}[SS:CUD:AAC].
Le diagramme de classes détaillé correspondant est à la \in{figure}[DC:AAC].

\item la \in{figure}[DS:RAC] correspond au \cu\ «RetirerArticleCadbot», voir \in{section}[SS:CUD:RAC].
Le diagramme de classes détaillé correspondant est à la \in{figure}[DC:RAC].


% TO BE CONTINUED
\stopitemize


\placefigure[][DS:BE]{DS du \cu\ «UtiliserBorneEuréka»}
{\externalfigure[ECLIPSE-LUNA-DS-BORNE-EUREKA/DSUtiliserBorneEureka.pdf][width=\hsize]}

\placefigure[][DS:AR1]{DS du \cu\ «AppliquerRecetteConnectée» (partie haute)}
{\externalfigure[ECLIPSE-LUNA-DS-APPLIQUER-RECETTE/DSAppliquerRecetteHAUT.pdf][orientation=90,height=.9\hsize]}
\placefigure[][DS:AR2]{DS du \cu\ «AppliquerRecetteConnectée» (partie basse)}
{\externalfigure[ECLIPSE-LUNA-DS-APPLIQUER-RECETTE/DSAppliquerRecetteBAS.pdf][orientation=90,height=1.1\hsize]}

\placefigure[][DS:AAC]{DS du \cu\ «AjouterArticleCadBot»}
{\externalfigure[ECLIPSE-LUNA-DS-AJOUTER-ARTICLE-CADBOT/DSAjouterArticleCadBot.pdf][width=\hsize]}

%\placefigure[][DS:AAC]{DS du \cu\ «RetirerArticleCadBot»}
%{\externalfigure[ECLIPSE-LUNA-DS-RETIRER-ARTICLE-CADBOT/RetirerArticleCadBot.pdf][width=\hsize]}

% TO BE CONTINUED




