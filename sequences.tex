\chapter{Diagrammes de séquences}

Dans cette section nous présentons les diagrammes de séquence (DS) reliés à quelques-uns des \cu\ les plus emblématiques de notre application.

\startitemize
\item la \in{figure}[DS:BE] correspond au \cu\ «UtiliserBorneEuréka», voir \in{section}[SS:CUD:UBE].
Le diagramme de classes détaillé correspondant est à la \in{figure}[DC:BE].
Les classes faisant partie de l'UI sont les «vues» et possèdent leur propre fil d'exécution afin de rester réactives aux sollicitations du client.
Par exemple, l'objet \type{carteUI} répond à des touchers de l'utilisateur pour faire défiler ou zoomer la carte (ces interactions ne sont pas représentées sur le diagramme).

\item la \in{figure}[DS:AR] correspond au \cu\ «AppliquerRecetteConnectée», voir \in{section}[SS:CUD:AR].
Le diagramme de classes détaillé correspondant est à la \in{figure}[DC:AR].

% TO BE CONTINUED
\stopitemize


\placefigure[][DS:BE]{DS du \cu\ «UtiliserBorneEuréka»}
{\externalfigure[ECLIPSE-LUNA-DS-BORNE-EUREKA/DSUtiliserBorneEureka.pdf][width=\hsize]}

\placefigure[][DS:AR]{DS du \cu\ «AppliquerRecetteConnectée»}
{\externalfigure[ECLIPSE-LUNA-DS-APPLIQUER-RECETTE/DSAppliquerRecette.pdf][width=\hsize]}

\placefigure[][DS:BE]{DS du \cu\ «UtiliserBorneEuréka»}
{\externalfigure[ECLIPSE-LUNA-SEQUENCES-AJOUTER-CADBOT-ARTICLE/DSAjouterArticleCadBot.pdf][width=\hsize]}



% TO BE CONTINUED




