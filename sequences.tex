\chapter{Diagrammes de séquences}

Dans cette section nous présentons les diagrammes de séquence (DS) reliés à quelques-uns des \cu\ les plus emblématiques de notre application.
\crlf
Ces diagrammes ont été élaborés selon le pattern Modèle-Vue-Contrôleur.
Étant donné qu'il existe plusieurs moutures différentes de ce pattern (qui diffèrent notamment par les relations entre le modèle et les vues), nous avons formulé notre propre synthèse que nous présentons pour référence et dans une version relativement détaillée, en annexe de ce document (\in{figure}[FIG:MVC]).
\par
Il faut noter également que nous parlons ici d'un modèle MVC «à gros grain» qui structure notre application (voir notamment \in{chapitre}[CH:ARCHI]). 
Ainsi, les objets de type «contrôle» correspondent aux principales actions définies dans les \cu.
Les objets «UI» sont des objets {\em frontière} complexes, qui seront eux-mêmes probablement implémentés en utilisant un deuxième modèle MVC à grains plus fins que l'on retrouve classiquement dans les constructions des interfaces utilisateurs.

\startitemize
\item la \in{figure}[DS:BE] correspond au \cu\ «UtiliserBorneEuréka», voir \in{section}[SS:CUD:UBE].
Le diagramme de classes détaillé correspondant est à la \in{figure}[DC:BE].
Les classes faisant partie de l'UI sont les «vues» et possèdent leur propre fil d'exécution afin de rester réactives aux sollicitations du client.
Par exemple, l'objet \type{carteUI} répond à des touchers de l'utilisateur pour faire défiler ou zoomer la carte (ces interactions ne sont pas représentées sur le diagramme).
Dans un souci de lisibilité, nous n'avons pas toujours indiqué les flèches de retour correspondant aux appels asynchrones.

\item la \in{figure}[DS:AAC] correspond au \cu\ «AjouterArticleCadbot», voir \in{section}[SS:CUD:AAC].
Le diagramme de classes détaillé correspondant est à la \in{figure}[DC:AAC].

\item la \in{figure}[DS:PP] correspond au \cu\ «PasserPortiquePaiement», voir \in{section}[SS:CUD:PP].
Le diagramme de classes détaillé correspondant est à la \in{figure}[DC:PP].

\item les \in{figures}[DS:AR1] et \in[DS:AR2] correspondent au \cu\ «AppliquerRecetteConnectée», voir \in{section}[SS:CUD:AR].
Le diagramme de classes détaillé correspondant est à la \in{figure}[DC:AR].
Ce diagramme de séquence ne reprend pas en intégralité l'ensemble des scénarios alternatifs du \cu; en effet cela aurait compliqué outre mesure le diagramme déjà trop chargé comme cela, sans rien apporter de neuf au niveau conception.

\item la \in{figure}[DS:GUIDAGE] correspond au \cu\ «UtiliserGuidageMarketTab» de la \in{section}[SS:GUIDAGE].
    %\TODO\ aucun objet modèle.. ??!!

\item les \in{figures}[DS:CLC] et \in[DS:DLC] correspondent aux \cu\ «Créer/DétruireRecetteConnectée» présentés à la \in{section}[SS:CLC] et \in[SS:DLC]. 
Le diagramme de classes correspondant est à la \in{figure}[DC:LC].
\stopitemize


\placefigure[][DS:BE]{DS du \cu\ «UtiliserBorneEuréka»}
{\externalfigure[ECLIPSE-LUNA-DS-BORNE-EUREKA/DSUtiliserBorneEureka.pdf][width=\hsize]}

\placefigure[][DS:AAC]{DS du \cu\ «AjouterArticleCadBot»}
{\externalfigure[ECLIPSE-LUNA-DS-AJOUTER-ARTICLE-CADBOT/DSAjouterArticleCadBot.pdf][width=\hsize]}

\placefigure[][DS:PP]{DS du \cu\ «PasserPortiquePaiement»}
{\externalfigure[ECLIPSE-LUNA-DS-PASSER-PORTIQUE/DSPasserPortiquePaiement.pdf][width=\hsize]}

\placefigure[][DS:AR1]{DS du \cu\ «AppliquerRecetteConnectée» (partie haute)}
{\externalfigure[ECLIPSE-LUNA-DS-APPLIQUER-RECETTE/DSAppliquerRecetteHAUT.pdf][orientation=90,width=23truecm,height=\hsize]}
\placefigure[][DS:AR2]{DS du \cu\ «AppliquerRecetteConnectée» (partie basse)}
{\externalfigure[ECLIPSE-LUNA-DS-APPLIQUER-RECETTE/DSAppliquerRecetteBAS.pdf][orientation=90,width=25truecm,height=\hsize]}

\placefigure[][DS:GUIDAGE]{DS du \cu\ «UtiliserGuidageMarkettab», auteur: Raïssa Mbabazi Simbi}
{\externalfigure[CROQUIS/DSguidage_markettab_new.svg][width=\hsize]}

\placefigure[][DS:CLC]{DS du \cu\ «CréerListeConnectée»}
{\externalfigure[CROQUIS/DS_CreerListeConnectee.pdf][width=.8\hsize]}
\placefigure[][DS:DLC]{DS du \cu\ «DétruireListeConnectée»}
{\externalfigure[CROQUIS/DS_DeleteListeConnectee.pdf][width=.8\hsize]}
% TO BE CONTINUED




